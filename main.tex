%% PREAMBLE %%
\documentclass[10pt, oneside]{book}
\usepackage{lipsum}
\usepackage{lmodern}
\usepackage[a4paper, margin=1in]{geometry}
\usepackage[]{fancyhdr}
\usepackage[]{titlesec}
\usepackage[dvipsnames]{xcolor}
\usepackage[hidelinks]{hyperref}
\usepackage{needspace}
\usepackage{enumitem}
\usepackage{layout}
\usepackage{tcolorbox}

%%---------------------- BASIC DOCUMENT DETAILS --------------------%%
%% These details are easiest hard-coded
\author{Anirudh Krishnan}
\title{Daily Journal 2023}
\date{\today}

%% Entire document is now in sans-serif
\renewcommand{\familydefault}{\sfdefault}

%%-------------- FINE CONTROL OVER DOCUMENT DIMENSIONS------------- %%
\setlength{\headsep}{54pt}
\setlength{\textheight}{680pt}
\setlength{\parskip}{1em}
\setlength{\marginparwidth}{60pt}
\setlength{\footskip}{30pt}

\setlist[itemize]{
  nosep,
  leftmargin=\parindent}

\setlist[description]{
  nosep,
  labelwidth=1in,
  labelsep = \parindent,
  font=\normalfont,
  align=right}


%%----------------- HIERARCHICAL SUBSUBSECTION NAMES ---------------%%

\let\Chaptermark\chaptermark
\let\Sectionmark\sectionmark
\let\Subsectionmark\subsectionmark
\let\Subsubsectionmark\subsubsectionmark
\def\chaptermark#1{
  \def\Chaptername{#1}
  \Chaptermark{#1}}
\def\sectionmark#1{
  \def\Sectionname{#1}
  \Sectionmark{#1}}
\def\subsectionmark#1{
  \def\Subsectionname{#1}
  \Subsectionmark{#1}}
\def\subsubsectionmark#1{
  \def\Subsubsectionname{#1}
  \Subsubsectionmark{#1}}


%%---------------SUPPRESSING SECTION, SUBSECTION TITLES ------------%%

\newcommand{\fakesection}[1]{%
  \par\refstepcounter{section}% Increase section counter
  \sectionmark{#1}% Add section mark (header)
  \addcontentsline
    {toc}
    {section}
    {\protect\numberline{\thesection}#1}% Add section to ToC
  %% Add more content here, if needed.
}

\newcommand{\fakesubsection}[1]{%
  \par\refstepcounter{subsection}% Increase subsection counter
  \subsectionmark{#1}% Add subsection mark (header)
  \addcontentsline
    {toc}
    {subsection}
    {\protect\numberline{\thesubsection}#1}% Add subsection to ToC
  %% Add more content here, if needed.
}


%%------------ CHAPTER, SECTION, SUBSECTION FOMRATTING -------------%%


\tcbset{
  size = small,
  on line,
  fontupper = {\strut},
  box align = base}

%% dont show chapter or section numbering (formally called labels) in
%% front of chapter or section names
\setcounter{secnumdepth}{-1}

%% dont show any subdivisions below chapters in TOC
\setcounter{tocdepth}{0}

%% Chapter formatting
\titleformat{\chapter}[block]
{\Huge}{}{0em}
{\tcbox[
  colframe=Goldenrod!50,
  colback=Goldenrod!20]}

%% Subsubsection heading using black text on colored boxes and
%% tcolorbox
\titleformat{\subsubsection}[block]{
  \tcbox[
    colframe=BrickRed!30,
    colback=BrickRed!10]{\Sectionname}
  \quad
  \tcbox[
    colframe=OliveGreen!30,
    colback=OliveGreen!10]{\Subsectionname}
  \quad}
  {}{0em}{
    \tcbox[
      colframe=MidnightBlue!30,
      colback=MidnightBlue!10]}[]

\titleformat{\paragraph}[block]
  {}{}{0em}{
    \tcbox[
    colframe=Black!30,
    colback=Black!10]}[]

%%------------------------- MAIN DOCUMENT --------------------------%%

\begin{document}
\layout
\begin{titlepage}
  \begin{titlepage}
\centering
	\settowidth{\unitlength}{\LARGE THE BOOK OF CONUNDRUMS}
	\vspace*{\baselineskip}
	{\large\scshape Anirudh Krishnan}\\[\baselineskip]
	\rule{\unitlength}{1.6pt}\vspace*{-\baselineskip}\vspace*{2pt}
	\rule{\unitlength}{0.4pt}\\[\baselineskip]
	{\LARGE Daily Journal}\\[\baselineskip]
	{\itshape Calendar Year 2023}\\[0.2\baselineskip]
	\rule{\unitlength}{0.4pt}\vspace*{-\baselineskip}\vspace{3.2pt}
	\rule{\unitlength}{1.6pt}\\[\baselineskip]
	\par
	\vfill
	{\large\scshape XYZ University}\\[\baselineskip]
	{\small\scshape Mars}\par
	\vspace*{0.1\textheight}
\end{titlepage}

  \frontmatter
\end{titlepage}

%% TOC entries line spacing reduced
\renewcommand{\baselinestretch}{0.25}\normalsize
\tableofcontents
\renewcommand{\baselinestretch}{1.0}\normalsize


\mainmatter

%%---------------- HEADER AND FOOTER FORMATTING --------------------%%

%% Set the pagestyle for plain pages which are on chapter start and
%% fancy pages used everywhere else
%% Once again the author and document title are hard-coded here for
%% convenience.
%% Defining and using variables like in a programming language
%% is too finicky
\fancypagestyle
  {plain}
  {\fancyhf{}
    \renewcommand{\headrulewidth}{0pt}
    \renewcommand{\footrulewidth}{0pt}}

\fancypagestyle
  {fancy}{
    \fancyhead[R]{Anirudh Krishnan}
    \fancyhead[L]{Journal 2023}
    \fancyfoot[]{}
    \renewcommand{\headrulewidth}{0pt}
    \renewcommand{\footrulewidth}{0pt}}

\pagestyle{fancy}



%%--------------------- CONTENT SECTION ----------------------------%%

%% Control for now is at the month level
%% For finer control, look at
%% './Diary_{year}/{Month_name}/full_summary.tex' which contains the
%% day wise text entries for work and leisure subsections
%% Uncomment the line corresponding to which kind of daily entries 
%% should be consolidated to create the yearly journal

% \IfFileExists{./Diary_2023/January/full_summary.tex}{\chapter{May}
	\section{Mon, May 01}
		\subsection{Meeting}
\lipsum[1-2]

		\subsection{Reading}
\lipsum[3-4]

	\section{Tue, May 02}
		\subsection{Meeting}
\lipsum[1-2]

		\subsection{Reading}
\lipsum[3-4]

	\section{Wed, May 03}
		\subsection{Meeting}
\lipsum[1-2]

		\subsection{Reading}
\lipsum[3-4]

	\section{Thu, May 04}
		\subsection{Meeting}
\lipsum[1-2]

		\subsection{Reading}
\lipsum[3-4]

	\section{Fri, May 05}
		\subsection{Meeting}
\lipsum[1-2]

		\subsection{Reading}
\lipsum[3-4]

	\section{Sat, May 06}
		\subsection{Meeting}
\lipsum[1-2]

		\subsection{Reading}
\lipsum[3-4]

	\section{Sun, May 07}
		\subsection{Meeting}
\lipsum[1-2]

		\subsection{Reading}
\lipsum[3-4]

	\section{Mon, May 08}
		\subsection{Meeting}
\lipsum[1-2]

		\subsection{Reading}
\lipsum[3-4]

	\section{Tue, May 09}
		\subsection{Meeting}
\lipsum[1-2]

		\subsection{Reading}
\lipsum[3-4]

	\section{Wed, May 10}
		\subsection{Meeting}
\lipsum[1-2]

		\subsection{Reading}
\lipsum[3-4]

	\section{Thu, May 11}
		\subsection{Meeting}
\lipsum[1-2]

		\subsection{Reading}
\lipsum[3-4]

	\section{Fri, May 12}
		\subsection{Meeting}
\lipsum[1-2]

		\subsection{Reading}
\lipsum[3-4]

	\section{Sat, May 13}
		\subsection{Meeting}
\lipsum[1-2]

		\subsection{Reading}
\lipsum[3-4]

	\section{Sun, May 14}
		\subsection{Meeting}
\lipsum[1-2]

		\subsection{Reading}
\lipsum[3-4]

	\section{Mon, May 15}
		\subsection{Meeting}
\lipsum[1-2]

		\subsection{Reading}
\lipsum[3-4]

	\section{Tue, May 16}
		\subsection{Meeting}
\lipsum[1-2]

		\subsection{Reading}
\lipsum[3-4]

	\section{Wed, May 17}
		\subsection{Meeting}
\lipsum[1-2]

		\subsection{Reading}
\lipsum[3-4]

	\section{Thu, May 18}
		\subsection{Meeting}
\lipsum[1-2]

		\subsection{Reading}
\lipsum[3-4]

	\section{Fri, May 19}
		\subsection{Meeting}
\lipsum[1-2]

		\subsection{Reading}
\lipsum[3-4]

	\section{Sat, May 20}
		\subsection{Meeting}
\lipsum[1-2]

		\subsection{Reading}
\lipsum[3-4]

	\section{Sun, May 21}
		\subsection{Meeting}
\lipsum[1-2]

		\subsection{Reading}
\lipsum[3-4]

	\section{Mon, May 22}
		\subsection{Meeting}
\lipsum[1-2]

		\subsection{Reading}
\lipsum[3-4]

	\section{Tue, May 23}
		\subsection{Meeting}
\lipsum[1-2]

		\subsection{Reading}
\lipsum[3-4]

	\section{Wed, May 24}
		\subsection{Meeting}
\lipsum[1-2]

		\subsection{Reading}
\lipsum[3-4]

	\section{Thu, May 25}
		\subsection{Meeting}
\lipsum[1-2]

		\subsection{Reading}
\lipsum[3-4]

	\section{Fri, May 26}
		\subsection{Meeting}
\lipsum[1-2]

		\subsection{Reading}
\lipsum[3-4]

	\section{Sat, May 27}
		\subsection{Meeting}
\lipsum[1-2]

		\subsection{Reading}
\lipsum[3-4]

	\section{Sun, May 28}
		\subsection{Meeting}
\lipsum[1-2]

		\subsection{Reading}
\lipsum[3-4]

	\section{Mon, May 29}
		\subsection{Meeting}
\lipsum[1-2]

		\subsection{Reading}
\lipsum[3-4]

	\section{Tue, May 30}
		\subsection{Meeting}
\lipsum[1-2]

		\subsection{Reading}
\lipsum[3-4]

	\section{Wed, May 31}
		\subsection{Meeting}
\lipsum[1-2]

		\subsection{Reading}
\lipsum[3-4]

}{}
\IfFileExists{./Diary_2023/February/full_summary.tex}{\chapter{May}
	\section{Mon, May 01}
		\subsection{Meeting}
\lipsum[1-2]

		\subsection{Reading}
\lipsum[3-4]

	\section{Tue, May 02}
		\subsection{Meeting}
\lipsum[1-2]

		\subsection{Reading}
\lipsum[3-4]

	\section{Wed, May 03}
		\subsection{Meeting}
\lipsum[1-2]

		\subsection{Reading}
\lipsum[3-4]

	\section{Thu, May 04}
		\subsection{Meeting}
\lipsum[1-2]

		\subsection{Reading}
\lipsum[3-4]

	\section{Fri, May 05}
		\subsection{Meeting}
\lipsum[1-2]

		\subsection{Reading}
\lipsum[3-4]

	\section{Sat, May 06}
		\subsection{Meeting}
\lipsum[1-2]

		\subsection{Reading}
\lipsum[3-4]

	\section{Sun, May 07}
		\subsection{Meeting}
\lipsum[1-2]

		\subsection{Reading}
\lipsum[3-4]

	\section{Mon, May 08}
		\subsection{Meeting}
\lipsum[1-2]

		\subsection{Reading}
\lipsum[3-4]

	\section{Tue, May 09}
		\subsection{Meeting}
\lipsum[1-2]

		\subsection{Reading}
\lipsum[3-4]

	\section{Wed, May 10}
		\subsection{Meeting}
\lipsum[1-2]

		\subsection{Reading}
\lipsum[3-4]

	\section{Thu, May 11}
		\subsection{Meeting}
\lipsum[1-2]

		\subsection{Reading}
\lipsum[3-4]

	\section{Fri, May 12}
		\subsection{Meeting}
\lipsum[1-2]

		\subsection{Reading}
\lipsum[3-4]

	\section{Sat, May 13}
		\subsection{Meeting}
\lipsum[1-2]

		\subsection{Reading}
\lipsum[3-4]

	\section{Sun, May 14}
		\subsection{Meeting}
\lipsum[1-2]

		\subsection{Reading}
\lipsum[3-4]

	\section{Mon, May 15}
		\subsection{Meeting}
\lipsum[1-2]

		\subsection{Reading}
\lipsum[3-4]

	\section{Tue, May 16}
		\subsection{Meeting}
\lipsum[1-2]

		\subsection{Reading}
\lipsum[3-4]

	\section{Wed, May 17}
		\subsection{Meeting}
\lipsum[1-2]

		\subsection{Reading}
\lipsum[3-4]

	\section{Thu, May 18}
		\subsection{Meeting}
\lipsum[1-2]

		\subsection{Reading}
\lipsum[3-4]

	\section{Fri, May 19}
		\subsection{Meeting}
\lipsum[1-2]

		\subsection{Reading}
\lipsum[3-4]

	\section{Sat, May 20}
		\subsection{Meeting}
\lipsum[1-2]

		\subsection{Reading}
\lipsum[3-4]

	\section{Sun, May 21}
		\subsection{Meeting}
\lipsum[1-2]

		\subsection{Reading}
\lipsum[3-4]

	\section{Mon, May 22}
		\subsection{Meeting}
\lipsum[1-2]

		\subsection{Reading}
\lipsum[3-4]

	\section{Tue, May 23}
		\subsection{Meeting}
\lipsum[1-2]

		\subsection{Reading}
\lipsum[3-4]

	\section{Wed, May 24}
		\subsection{Meeting}
\lipsum[1-2]

		\subsection{Reading}
\lipsum[3-4]

	\section{Thu, May 25}
		\subsection{Meeting}
\lipsum[1-2]

		\subsection{Reading}
\lipsum[3-4]

	\section{Fri, May 26}
		\subsection{Meeting}
\lipsum[1-2]

		\subsection{Reading}
\lipsum[3-4]

	\section{Sat, May 27}
		\subsection{Meeting}
\lipsum[1-2]

		\subsection{Reading}
\lipsum[3-4]

	\section{Sun, May 28}
		\subsection{Meeting}
\lipsum[1-2]

		\subsection{Reading}
\lipsum[3-4]

	\section{Mon, May 29}
		\subsection{Meeting}
\lipsum[1-2]

		\subsection{Reading}
\lipsum[3-4]

	\section{Tue, May 30}
		\subsection{Meeting}
\lipsum[1-2]

		\subsection{Reading}
\lipsum[3-4]

	\section{Wed, May 31}
		\subsection{Meeting}
\lipsum[1-2]

		\subsection{Reading}
\lipsum[3-4]

}{}
\IfFileExists{./Diary_2023/March/full_summary.tex}{\chapter{May}
	\section{Mon, May 01}
		\subsection{Meeting}
\lipsum[1-2]

		\subsection{Reading}
\lipsum[3-4]

	\section{Tue, May 02}
		\subsection{Meeting}
\lipsum[1-2]

		\subsection{Reading}
\lipsum[3-4]

	\section{Wed, May 03}
		\subsection{Meeting}
\lipsum[1-2]

		\subsection{Reading}
\lipsum[3-4]

	\section{Thu, May 04}
		\subsection{Meeting}
\lipsum[1-2]

		\subsection{Reading}
\lipsum[3-4]

	\section{Fri, May 05}
		\subsection{Meeting}
\lipsum[1-2]

		\subsection{Reading}
\lipsum[3-4]

	\section{Sat, May 06}
		\subsection{Meeting}
\lipsum[1-2]

		\subsection{Reading}
\lipsum[3-4]

	\section{Sun, May 07}
		\subsection{Meeting}
\lipsum[1-2]

		\subsection{Reading}
\lipsum[3-4]

	\section{Mon, May 08}
		\subsection{Meeting}
\lipsum[1-2]

		\subsection{Reading}
\lipsum[3-4]

	\section{Tue, May 09}
		\subsection{Meeting}
\lipsum[1-2]

		\subsection{Reading}
\lipsum[3-4]

	\section{Wed, May 10}
		\subsection{Meeting}
\lipsum[1-2]

		\subsection{Reading}
\lipsum[3-4]

	\section{Thu, May 11}
		\subsection{Meeting}
\lipsum[1-2]

		\subsection{Reading}
\lipsum[3-4]

	\section{Fri, May 12}
		\subsection{Meeting}
\lipsum[1-2]

		\subsection{Reading}
\lipsum[3-4]

	\section{Sat, May 13}
		\subsection{Meeting}
\lipsum[1-2]

		\subsection{Reading}
\lipsum[3-4]

	\section{Sun, May 14}
		\subsection{Meeting}
\lipsum[1-2]

		\subsection{Reading}
\lipsum[3-4]

	\section{Mon, May 15}
		\subsection{Meeting}
\lipsum[1-2]

		\subsection{Reading}
\lipsum[3-4]

	\section{Tue, May 16}
		\subsection{Meeting}
\lipsum[1-2]

		\subsection{Reading}
\lipsum[3-4]

	\section{Wed, May 17}
		\subsection{Meeting}
\lipsum[1-2]

		\subsection{Reading}
\lipsum[3-4]

	\section{Thu, May 18}
		\subsection{Meeting}
\lipsum[1-2]

		\subsection{Reading}
\lipsum[3-4]

	\section{Fri, May 19}
		\subsection{Meeting}
\lipsum[1-2]

		\subsection{Reading}
\lipsum[3-4]

	\section{Sat, May 20}
		\subsection{Meeting}
\lipsum[1-2]

		\subsection{Reading}
\lipsum[3-4]

	\section{Sun, May 21}
		\subsection{Meeting}
\lipsum[1-2]

		\subsection{Reading}
\lipsum[3-4]

	\section{Mon, May 22}
		\subsection{Meeting}
\lipsum[1-2]

		\subsection{Reading}
\lipsum[3-4]

	\section{Tue, May 23}
		\subsection{Meeting}
\lipsum[1-2]

		\subsection{Reading}
\lipsum[3-4]

	\section{Wed, May 24}
		\subsection{Meeting}
\lipsum[1-2]

		\subsection{Reading}
\lipsum[3-4]

	\section{Thu, May 25}
		\subsection{Meeting}
\lipsum[1-2]

		\subsection{Reading}
\lipsum[3-4]

	\section{Fri, May 26}
		\subsection{Meeting}
\lipsum[1-2]

		\subsection{Reading}
\lipsum[3-4]

	\section{Sat, May 27}
		\subsection{Meeting}
\lipsum[1-2]

		\subsection{Reading}
\lipsum[3-4]

	\section{Sun, May 28}
		\subsection{Meeting}
\lipsum[1-2]

		\subsection{Reading}
\lipsum[3-4]

	\section{Mon, May 29}
		\subsection{Meeting}
\lipsum[1-2]

		\subsection{Reading}
\lipsum[3-4]

	\section{Tue, May 30}
		\subsection{Meeting}
\lipsum[1-2]

		\subsection{Reading}
\lipsum[3-4]

	\section{Wed, May 31}
		\subsection{Meeting}
\lipsum[1-2]

		\subsection{Reading}
\lipsum[3-4]

}{}
% \IfFileExists{./Diary_2023/April/full_summary.tex}{\chapter{April}
	\section{Sat, Apr 01}
		\subsection{Meeting}
\lipsum[1-2]

		\subsection{Reading}
\lipsum[3-4]

	\section{Sun, Apr 02}
		\subsection{Meeting}
\lipsum[1-2]

		\subsection{Reading}
\lipsum[3-4]

	\section{Mon, Apr 03}
		\subsection{Meeting}
\lipsum[1-2]

		\subsection{Reading}
\lipsum[3-4]

	\section{Tue, Apr 04}
		\subsection{Meeting}
\lipsum[1-2]

		\subsection{Reading}
\lipsum[3-4]

	\section{Wed, Apr 05}
		\subsection{Meeting}
\lipsum[1-2]

		\subsection{Reading}
\lipsum[3-4]

	\section{Thu, Apr 06}
		\subsection{Meeting}
\lipsum[1-2]

		\subsection{Reading}
\lipsum[3-4]

	\section{Fri, Apr 07}
		\subsection{Meeting}
\lipsum[1-2]

		\subsection{Reading}
\lipsum[3-4]

	\section{Sat, Apr 08}
		\subsection{Meeting}
\lipsum[1-2]

		\subsection{Reading}
\lipsum[3-4]

	\section{Sun, Apr 09}
		\subsection{Meeting}
\lipsum[1-2]

		\subsection{Reading}
\lipsum[3-4]

	\section{Mon, Apr 10}
		\subsection{Meeting}
\lipsum[1-2]

		\subsection{Reading}
\lipsum[3-4]

	\section{Tue, Apr 11}
		\subsection{Meeting}
\lipsum[1-2]

		\subsection{Reading}
\lipsum[3-4]

	\section{Wed, Apr 12}
		\subsection{Meeting}
\lipsum[1-2]

		\subsection{Reading}
\lipsum[3-4]

	\section{Thu, Apr 13}
		\subsection{Meeting}
\lipsum[1-2]

		\subsection{Reading}
\lipsum[3-4]

	\section{Fri, Apr 14}
		\subsection{Meeting}
\lipsum[1-2]

		\subsection{Reading}
\lipsum[3-4]

	\section{Sat, Apr 15}
		\subsection{Meeting}
\lipsum[1-2]

		\subsection{Reading}
\lipsum[3-4]

	\section{Sun, Apr 16}
		\subsection{Meeting}
\lipsum[1-2]

		\subsection{Reading}
\lipsum[3-4]

	\section{Mon, Apr 17}
		\subsection{Meeting}
\lipsum[1-2]

		\subsection{Reading}
\lipsum[3-4]

	\section{Tue, Apr 18}
		\subsection{Meeting}
\lipsum[1-2]

		\subsection{Reading}
\lipsum[3-4]

	\section{Wed, Apr 19}
		\subsection{Meeting}
\lipsum[1-2]

		\subsection{Reading}
\lipsum[3-4]

	\section{Thu, Apr 20}
		\subsection{Meeting}
\lipsum[1-2]

		\subsection{Reading}
\lipsum[3-4]

	\section{Fri, Apr 21}
		\subsection{Meeting}
\lipsum[1-2]

		\subsection{Reading}
\lipsum[3-4]

	\section{Sat, Apr 22}
		\subsection{Meeting}
\lipsum[1-2]

		\subsection{Reading}
\lipsum[3-4]

	\section{Sun, Apr 23}
		\subsection{Meeting}
\lipsum[1-2]

		\subsection{Reading}
\lipsum[3-4]

	\section{Mon, Apr 24}
		\subsection{Meeting}
\lipsum[1-2]

		\subsection{Reading}
\lipsum[3-4]

	\section{Tue, Apr 25}
		\subsection{Meeting}
\lipsum[1-2]

		\subsection{Reading}
\lipsum[3-4]

	\section{Wed, Apr 26}
		\subsection{Meeting}
\lipsum[1-2]

		\subsection{Reading}
\lipsum[3-4]

	\section{Thu, Apr 27}
		\subsection{Meeting}
\lipsum[1-2]

		\subsection{Reading}
\lipsum[3-4]

	\section{Fri, Apr 28}
		\subsection{Meeting}
\lipsum[1-2]

		\subsection{Reading}
\lipsum[3-4]

	\section{Sat, Apr 29}
		\subsection{Meeting}
\lipsum[1-2]

		\subsection{Reading}
\lipsum[3-4]

	\section{Sun, Apr 30}
		\subsection{Meeting}
\lipsum[1-2]

		\subsection{Reading}
\lipsum[3-4]

}{}
% \IfFileExists{./Diary_2023/May/full_summary.tex}{\chapter{May}
	\section{Mon, May 01}
		\subsection{Meeting}
\lipsum[1-2]

		\subsection{Reading}
\lipsum[3-4]

	\section{Tue, May 02}
		\subsection{Meeting}
\lipsum[1-2]

		\subsection{Reading}
\lipsum[3-4]

	\section{Wed, May 03}
		\subsection{Meeting}
\lipsum[1-2]

		\subsection{Reading}
\lipsum[3-4]

	\section{Thu, May 04}
		\subsection{Meeting}
\lipsum[1-2]

		\subsection{Reading}
\lipsum[3-4]

	\section{Fri, May 05}
		\subsection{Meeting}
\lipsum[1-2]

		\subsection{Reading}
\lipsum[3-4]

	\section{Sat, May 06}
		\subsection{Meeting}
\lipsum[1-2]

		\subsection{Reading}
\lipsum[3-4]

	\section{Sun, May 07}
		\subsection{Meeting}
\lipsum[1-2]

		\subsection{Reading}
\lipsum[3-4]

	\section{Mon, May 08}
		\subsection{Meeting}
\lipsum[1-2]

		\subsection{Reading}
\lipsum[3-4]

	\section{Tue, May 09}
		\subsection{Meeting}
\lipsum[1-2]

		\subsection{Reading}
\lipsum[3-4]

	\section{Wed, May 10}
		\subsection{Meeting}
\lipsum[1-2]

		\subsection{Reading}
\lipsum[3-4]

	\section{Thu, May 11}
		\subsection{Meeting}
\lipsum[1-2]

		\subsection{Reading}
\lipsum[3-4]

	\section{Fri, May 12}
		\subsection{Meeting}
\lipsum[1-2]

		\subsection{Reading}
\lipsum[3-4]

	\section{Sat, May 13}
		\subsection{Meeting}
\lipsum[1-2]

		\subsection{Reading}
\lipsum[3-4]

	\section{Sun, May 14}
		\subsection{Meeting}
\lipsum[1-2]

		\subsection{Reading}
\lipsum[3-4]

	\section{Mon, May 15}
		\subsection{Meeting}
\lipsum[1-2]

		\subsection{Reading}
\lipsum[3-4]

	\section{Tue, May 16}
		\subsection{Meeting}
\lipsum[1-2]

		\subsection{Reading}
\lipsum[3-4]

	\section{Wed, May 17}
		\subsection{Meeting}
\lipsum[1-2]

		\subsection{Reading}
\lipsum[3-4]

	\section{Thu, May 18}
		\subsection{Meeting}
\lipsum[1-2]

		\subsection{Reading}
\lipsum[3-4]

	\section{Fri, May 19}
		\subsection{Meeting}
\lipsum[1-2]

		\subsection{Reading}
\lipsum[3-4]

	\section{Sat, May 20}
		\subsection{Meeting}
\lipsum[1-2]

		\subsection{Reading}
\lipsum[3-4]

	\section{Sun, May 21}
		\subsection{Meeting}
\lipsum[1-2]

		\subsection{Reading}
\lipsum[3-4]

	\section{Mon, May 22}
		\subsection{Meeting}
\lipsum[1-2]

		\subsection{Reading}
\lipsum[3-4]

	\section{Tue, May 23}
		\subsection{Meeting}
\lipsum[1-2]

		\subsection{Reading}
\lipsum[3-4]

	\section{Wed, May 24}
		\subsection{Meeting}
\lipsum[1-2]

		\subsection{Reading}
\lipsum[3-4]

	\section{Thu, May 25}
		\subsection{Meeting}
\lipsum[1-2]

		\subsection{Reading}
\lipsum[3-4]

	\section{Fri, May 26}
		\subsection{Meeting}
\lipsum[1-2]

		\subsection{Reading}
\lipsum[3-4]

	\section{Sat, May 27}
		\subsection{Meeting}
\lipsum[1-2]

		\subsection{Reading}
\lipsum[3-4]

	\section{Sun, May 28}
		\subsection{Meeting}
\lipsum[1-2]

		\subsection{Reading}
\lipsum[3-4]

	\section{Mon, May 29}
		\subsection{Meeting}
\lipsum[1-2]

		\subsection{Reading}
\lipsum[3-4]

	\section{Tue, May 30}
		\subsection{Meeting}
\lipsum[1-2]

		\subsection{Reading}
\lipsum[3-4]

	\section{Wed, May 31}
		\subsection{Meeting}
\lipsum[1-2]

		\subsection{Reading}
\lipsum[3-4]

}{}
% \IfFileExists{./Diary_2023/June/full_summary.tex}{\chapter{May}
	\section{Mon, May 01}
		\subsection{Meeting}
\lipsum[1-2]

		\subsection{Reading}
\lipsum[3-4]

	\section{Tue, May 02}
		\subsection{Meeting}
\lipsum[1-2]

		\subsection{Reading}
\lipsum[3-4]

	\section{Wed, May 03}
		\subsection{Meeting}
\lipsum[1-2]

		\subsection{Reading}
\lipsum[3-4]

	\section{Thu, May 04}
		\subsection{Meeting}
\lipsum[1-2]

		\subsection{Reading}
\lipsum[3-4]

	\section{Fri, May 05}
		\subsection{Meeting}
\lipsum[1-2]

		\subsection{Reading}
\lipsum[3-4]

	\section{Sat, May 06}
		\subsection{Meeting}
\lipsum[1-2]

		\subsection{Reading}
\lipsum[3-4]

	\section{Sun, May 07}
		\subsection{Meeting}
\lipsum[1-2]

		\subsection{Reading}
\lipsum[3-4]

	\section{Mon, May 08}
		\subsection{Meeting}
\lipsum[1-2]

		\subsection{Reading}
\lipsum[3-4]

	\section{Tue, May 09}
		\subsection{Meeting}
\lipsum[1-2]

		\subsection{Reading}
\lipsum[3-4]

	\section{Wed, May 10}
		\subsection{Meeting}
\lipsum[1-2]

		\subsection{Reading}
\lipsum[3-4]

	\section{Thu, May 11}
		\subsection{Meeting}
\lipsum[1-2]

		\subsection{Reading}
\lipsum[3-4]

	\section{Fri, May 12}
		\subsection{Meeting}
\lipsum[1-2]

		\subsection{Reading}
\lipsum[3-4]

	\section{Sat, May 13}
		\subsection{Meeting}
\lipsum[1-2]

		\subsection{Reading}
\lipsum[3-4]

	\section{Sun, May 14}
		\subsection{Meeting}
\lipsum[1-2]

		\subsection{Reading}
\lipsum[3-4]

	\section{Mon, May 15}
		\subsection{Meeting}
\lipsum[1-2]

		\subsection{Reading}
\lipsum[3-4]

	\section{Tue, May 16}
		\subsection{Meeting}
\lipsum[1-2]

		\subsection{Reading}
\lipsum[3-4]

	\section{Wed, May 17}
		\subsection{Meeting}
\lipsum[1-2]

		\subsection{Reading}
\lipsum[3-4]

	\section{Thu, May 18}
		\subsection{Meeting}
\lipsum[1-2]

		\subsection{Reading}
\lipsum[3-4]

	\section{Fri, May 19}
		\subsection{Meeting}
\lipsum[1-2]

		\subsection{Reading}
\lipsum[3-4]

	\section{Sat, May 20}
		\subsection{Meeting}
\lipsum[1-2]

		\subsection{Reading}
\lipsum[3-4]

	\section{Sun, May 21}
		\subsection{Meeting}
\lipsum[1-2]

		\subsection{Reading}
\lipsum[3-4]

	\section{Mon, May 22}
		\subsection{Meeting}
\lipsum[1-2]

		\subsection{Reading}
\lipsum[3-4]

	\section{Tue, May 23}
		\subsection{Meeting}
\lipsum[1-2]

		\subsection{Reading}
\lipsum[3-4]

	\section{Wed, May 24}
		\subsection{Meeting}
\lipsum[1-2]

		\subsection{Reading}
\lipsum[3-4]

	\section{Thu, May 25}
		\subsection{Meeting}
\lipsum[1-2]

		\subsection{Reading}
\lipsum[3-4]

	\section{Fri, May 26}
		\subsection{Meeting}
\lipsum[1-2]

		\subsection{Reading}
\lipsum[3-4]

	\section{Sat, May 27}
		\subsection{Meeting}
\lipsum[1-2]

		\subsection{Reading}
\lipsum[3-4]

	\section{Sun, May 28}
		\subsection{Meeting}
\lipsum[1-2]

		\subsection{Reading}
\lipsum[3-4]

	\section{Mon, May 29}
		\subsection{Meeting}
\lipsum[1-2]

		\subsection{Reading}
\lipsum[3-4]

	\section{Tue, May 30}
		\subsection{Meeting}
\lipsum[1-2]

		\subsection{Reading}
\lipsum[3-4]

	\section{Wed, May 31}
		\subsection{Meeting}
\lipsum[1-2]

		\subsection{Reading}
\lipsum[3-4]

}{}
% \IfFileExists{./Diary_2023/July/full_summary.tex}{\chapter{July}
	\section{Sat, Jul 01}
		\subsection{Meeting}
\lipsum[1-2]

		\subsection{Reading}
\lipsum[3-4]

	\section{Sun, Jul 02}
		\subsection{Meeting}
\lipsum[1-2]

		\subsection{Reading}
\lipsum[3-4]

	\section{Mon, Jul 03}
		\subsection{Meeting}
\lipsum[1-2]

		\subsection{Reading}
\lipsum[3-4]

	\section{Tue, Jul 04}
		\subsection{Meeting}
\lipsum[1-2]

		\subsection{Reading}
\lipsum[3-4]

	\section{Wed, Jul 05}
		\subsection{Meeting}
\lipsum[1-2]

		\subsection{Reading}
\lipsum[3-4]

	\section{Thu, Jul 06}
		\subsection{Meeting}
\lipsum[1-2]

		\subsection{Reading}
\lipsum[3-4]

	\section{Fri, Jul 07}
		\subsection{Meeting}
\lipsum[1-2]

		\subsection{Reading}
\lipsum[3-4]

	\section{Sat, Jul 08}
		\subsection{Meeting}
\lipsum[1-2]

		\subsection{Reading}
\lipsum[3-4]

	\section{Sun, Jul 09}
		\subsection{Meeting}
\lipsum[1-2]

		\subsection{Reading}
\lipsum[3-4]

	\section{Mon, Jul 10}
		\subsection{Meeting}
\lipsum[1-2]

		\subsection{Reading}
\lipsum[3-4]

	\section{Tue, Jul 11}
		\subsection{Meeting}
\lipsum[1-2]

		\subsection{Reading}
\lipsum[3-4]

	\section{Wed, Jul 12}
		\subsection{Meeting}
\lipsum[1-2]

		\subsection{Reading}
\lipsum[3-4]

	\section{Thu, Jul 13}
		\subsection{Meeting}
\lipsum[1-2]

		\subsection{Reading}
\lipsum[3-4]

	\section{Fri, Jul 14}
		\subsection{Meeting}
\lipsum[1-2]

		\subsection{Reading}
\lipsum[3-4]

	\section{Sat, Jul 15}
		\subsection{Meeting}
\lipsum[1-2]

		\subsection{Reading}
\lipsum[3-4]

	\section{Sun, Jul 16}
		\subsection{Meeting}
\lipsum[1-2]

		\subsection{Reading}
\lipsum[3-4]

	\section{Mon, Jul 17}
		\subsection{Meeting}
\lipsum[1-2]

		\subsection{Reading}
\lipsum[3-4]

	\section{Tue, Jul 18}
		\subsection{Meeting}
\lipsum[1-2]

		\subsection{Reading}
\lipsum[3-4]

	\section{Wed, Jul 19}
		\subsection{Meeting}
\lipsum[1-2]

		\subsection{Reading}
\lipsum[3-4]

	\section{Thu, Jul 20}
		\subsection{Meeting}
\lipsum[1-2]

		\subsection{Reading}
\lipsum[3-4]

	\section{Fri, Jul 21}
		\subsection{Meeting}
\lipsum[1-2]

		\subsection{Reading}
\lipsum[3-4]

	\section{Sat, Jul 22}
		\subsection{Meeting}
\lipsum[1-2]

		\subsection{Reading}
\lipsum[3-4]

	\section{Sun, Jul 23}
		\subsection{Meeting}
\lipsum[1-2]

		\subsection{Reading}
\lipsum[3-4]

	\section{Mon, Jul 24}
		\subsection{Meeting}
\lipsum[1-2]

		\subsection{Reading}
\lipsum[3-4]

	\section{Tue, Jul 25}
		\subsection{Meeting}
\lipsum[1-2]

		\subsection{Reading}
\lipsum[3-4]

	\section{Wed, Jul 26}
		\subsection{Meeting}
\lipsum[1-2]

		\subsection{Reading}
\lipsum[3-4]

	\section{Thu, Jul 27}
		\subsection{Meeting}
\lipsum[1-2]

		\subsection{Reading}
\lipsum[3-4]

	\section{Fri, Jul 28}
		\subsection{Meeting}
\lipsum[1-2]

		\subsection{Reading}
\lipsum[3-4]

	\section{Sat, Jul 29}
		\subsection{Meeting}
\lipsum[1-2]

		\subsection{Reading}
\lipsum[3-4]

	\section{Sun, Jul 30}
		\subsection{Meeting}
\lipsum[1-2]

		\subsection{Reading}
\lipsum[3-4]

	\section{Mon, Jul 31}
		\subsection{Meeting}
\lipsum[1-2]

		\subsection{Reading}
\lipsum[3-4]

}{}
% \IfFileExists{./Diary_2023/August/full_summary.tex}{\chapter{May}
	\section{Mon, May 01}
		\subsection{Meeting}
\lipsum[1-2]

		\subsection{Reading}
\lipsum[3-4]

	\section{Tue, May 02}
		\subsection{Meeting}
\lipsum[1-2]

		\subsection{Reading}
\lipsum[3-4]

	\section{Wed, May 03}
		\subsection{Meeting}
\lipsum[1-2]

		\subsection{Reading}
\lipsum[3-4]

	\section{Thu, May 04}
		\subsection{Meeting}
\lipsum[1-2]

		\subsection{Reading}
\lipsum[3-4]

	\section{Fri, May 05}
		\subsection{Meeting}
\lipsum[1-2]

		\subsection{Reading}
\lipsum[3-4]

	\section{Sat, May 06}
		\subsection{Meeting}
\lipsum[1-2]

		\subsection{Reading}
\lipsum[3-4]

	\section{Sun, May 07}
		\subsection{Meeting}
\lipsum[1-2]

		\subsection{Reading}
\lipsum[3-4]

	\section{Mon, May 08}
		\subsection{Meeting}
\lipsum[1-2]

		\subsection{Reading}
\lipsum[3-4]

	\section{Tue, May 09}
		\subsection{Meeting}
\lipsum[1-2]

		\subsection{Reading}
\lipsum[3-4]

	\section{Wed, May 10}
		\subsection{Meeting}
\lipsum[1-2]

		\subsection{Reading}
\lipsum[3-4]

	\section{Thu, May 11}
		\subsection{Meeting}
\lipsum[1-2]

		\subsection{Reading}
\lipsum[3-4]

	\section{Fri, May 12}
		\subsection{Meeting}
\lipsum[1-2]

		\subsection{Reading}
\lipsum[3-4]

	\section{Sat, May 13}
		\subsection{Meeting}
\lipsum[1-2]

		\subsection{Reading}
\lipsum[3-4]

	\section{Sun, May 14}
		\subsection{Meeting}
\lipsum[1-2]

		\subsection{Reading}
\lipsum[3-4]

	\section{Mon, May 15}
		\subsection{Meeting}
\lipsum[1-2]

		\subsection{Reading}
\lipsum[3-4]

	\section{Tue, May 16}
		\subsection{Meeting}
\lipsum[1-2]

		\subsection{Reading}
\lipsum[3-4]

	\section{Wed, May 17}
		\subsection{Meeting}
\lipsum[1-2]

		\subsection{Reading}
\lipsum[3-4]

	\section{Thu, May 18}
		\subsection{Meeting}
\lipsum[1-2]

		\subsection{Reading}
\lipsum[3-4]

	\section{Fri, May 19}
		\subsection{Meeting}
\lipsum[1-2]

		\subsection{Reading}
\lipsum[3-4]

	\section{Sat, May 20}
		\subsection{Meeting}
\lipsum[1-2]

		\subsection{Reading}
\lipsum[3-4]

	\section{Sun, May 21}
		\subsection{Meeting}
\lipsum[1-2]

		\subsection{Reading}
\lipsum[3-4]

	\section{Mon, May 22}
		\subsection{Meeting}
\lipsum[1-2]

		\subsection{Reading}
\lipsum[3-4]

	\section{Tue, May 23}
		\subsection{Meeting}
\lipsum[1-2]

		\subsection{Reading}
\lipsum[3-4]

	\section{Wed, May 24}
		\subsection{Meeting}
\lipsum[1-2]

		\subsection{Reading}
\lipsum[3-4]

	\section{Thu, May 25}
		\subsection{Meeting}
\lipsum[1-2]

		\subsection{Reading}
\lipsum[3-4]

	\section{Fri, May 26}
		\subsection{Meeting}
\lipsum[1-2]

		\subsection{Reading}
\lipsum[3-4]

	\section{Sat, May 27}
		\subsection{Meeting}
\lipsum[1-2]

		\subsection{Reading}
\lipsum[3-4]

	\section{Sun, May 28}
		\subsection{Meeting}
\lipsum[1-2]

		\subsection{Reading}
\lipsum[3-4]

	\section{Mon, May 29}
		\subsection{Meeting}
\lipsum[1-2]

		\subsection{Reading}
\lipsum[3-4]

	\section{Tue, May 30}
		\subsection{Meeting}
\lipsum[1-2]

		\subsection{Reading}
\lipsum[3-4]

	\section{Wed, May 31}
		\subsection{Meeting}
\lipsum[1-2]

		\subsection{Reading}
\lipsum[3-4]

}{}
% \IfFileExists{./Diary_2023/September/full_summary.tex}{\chapter{September}
	\section{Fri, Sep 01}
		\subsection{Meeting}
\lipsum[1-2]

		\subsection{Reading}
\lipsum[3-4]

	\section{Sat, Sep 02}
		\subsection{Meeting}
\lipsum[1-2]

		\subsection{Reading}
\lipsum[3-4]

	\section{Sun, Sep 03}
		\subsection{Meeting}
\lipsum[1-2]

		\subsection{Reading}
\lipsum[3-4]

	\section{Mon, Sep 04}
		\subsection{Meeting}
\lipsum[1-2]

		\subsection{Reading}
\lipsum[3-4]

	\section{Tue, Sep 05}
		\subsection{Meeting}
\lipsum[1-2]

		\subsection{Reading}
\lipsum[3-4]

	\section{Wed, Sep 06}
		\subsection{Meeting}
\lipsum[1-2]

		\subsection{Reading}
\lipsum[3-4]

	\section{Thu, Sep 07}
		\subsection{Meeting}
\lipsum[1-2]

		\subsection{Reading}
\lipsum[3-4]

	\section{Fri, Sep 08}
		\subsection{Meeting}
\lipsum[1-2]

		\subsection{Reading}
\lipsum[3-4]

	\section{Sat, Sep 09}
		\subsection{Meeting}
\lipsum[1-2]

		\subsection{Reading}
\lipsum[3-4]

	\section{Sun, Sep 10}
		\subsection{Meeting}
\lipsum[1-2]

		\subsection{Reading}
\lipsum[3-4]

	\section{Mon, Sep 11}
		\subsection{Meeting}
\lipsum[1-2]

		\subsection{Reading}
\lipsum[3-4]

	\section{Tue, Sep 12}
		\subsection{Meeting}
\lipsum[1-2]

		\subsection{Reading}
\lipsum[3-4]

	\section{Wed, Sep 13}
		\subsection{Meeting}
\lipsum[1-2]

		\subsection{Reading}
\lipsum[3-4]

	\section{Thu, Sep 14}
		\subsection{Meeting}
\lipsum[1-2]

		\subsection{Reading}
\lipsum[3-4]

	\section{Fri, Sep 15}
		\subsection{Meeting}
\lipsum[1-2]

		\subsection{Reading}
\lipsum[3-4]

	\section{Sat, Sep 16}
		\subsection{Meeting}
\lipsum[1-2]

		\subsection{Reading}
\lipsum[3-4]

	\section{Sun, Sep 17}
		\subsection{Meeting}
\lipsum[1-2]

		\subsection{Reading}
\lipsum[3-4]

	\section{Mon, Sep 18}
		\subsection{Meeting}
\lipsum[1-2]

		\subsection{Reading}
\lipsum[3-4]

	\section{Tue, Sep 19}
		\subsection{Meeting}
\lipsum[1-2]

		\subsection{Reading}
\lipsum[3-4]

	\section{Wed, Sep 20}
		\subsection{Meeting}
\lipsum[1-2]

		\subsection{Reading}
\lipsum[3-4]

	\section{Thu, Sep 21}
		\subsection{Meeting}
\lipsum[1-2]

		\subsection{Reading}
\lipsum[3-4]

	\section{Fri, Sep 22}
		\subsection{Meeting}
\lipsum[1-2]

		\subsection{Reading}
\lipsum[3-4]

	\section{Sat, Sep 23}
		\subsection{Meeting}
\lipsum[1-2]

		\subsection{Reading}
\lipsum[3-4]

	\section{Sun, Sep 24}
		\subsection{Meeting}
\lipsum[1-2]

		\subsection{Reading}
\lipsum[3-4]

	\section{Mon, Sep 25}
		\subsection{Meeting}
\lipsum[1-2]

		\subsection{Reading}
\lipsum[3-4]

	\section{Tue, Sep 26}
		\subsection{Meeting}
\lipsum[1-2]

		\subsection{Reading}
\lipsum[3-4]

	\section{Wed, Sep 27}
		\subsection{Meeting}
\lipsum[1-2]

		\subsection{Reading}
\lipsum[3-4]

	\section{Thu, Sep 28}
		\subsection{Meeting}
\lipsum[1-2]

		\subsection{Reading}
\lipsum[3-4]

	\section{Fri, Sep 29}
		\subsection{Meeting}
\lipsum[1-2]

		\subsection{Reading}
\lipsum[3-4]

	\section{Sat, Sep 30}
		\subsection{Meeting}
\lipsum[1-2]

		\subsection{Reading}
\lipsum[3-4]

}{}
% \IfFileExists{./Diary_2023/October/full_summary.tex}{\chapter{October}
	\Needspace{10\baselineskip}
	\section{Sun, Oct 01}
		\subsection{Meeting}
\lipsum[1-2]

		\subsection{Reading}
\lipsum[3-4]

	\Needspace{10\baselineskip}
	\section{Mon, Oct 02}
		\subsection{Meeting}
\lipsum[1-2]

		\subsection{Reading}
\lipsum[3-4]

	\Needspace{10\baselineskip}
	\section{Tue, Oct 03}
		\subsection{Meeting}
\lipsum[1-2]

		\subsection{Reading}
\lipsum[3-4]

	\Needspace{10\baselineskip}
	\section{Wed, Oct 04}
		\subsection{Meeting}
\lipsum[1-2]

		\subsection{Reading}
\lipsum[3-4]

	\Needspace{10\baselineskip}
	\section{Thu, Oct 05}
		\subsection{Meeting}
\lipsum[1-2]

		\subsection{Reading}
\lipsum[3-4]

	\Needspace{10\baselineskip}
	\section{Fri, Oct 06}
		\subsection{Meeting}
\lipsum[1-2]

		\subsection{Reading}
\lipsum[3-4]

	\Needspace{10\baselineskip}
	\section{Sat, Oct 07}
		\subsection{Meeting}
\lipsum[1-2]

		\subsection{Reading}
\lipsum[3-4]

	\Needspace{10\baselineskip}
	\section{Sun, Oct 08}
		\subsection{Meeting}
\lipsum[1-2]

		\subsection{Reading}
\lipsum[3-4]

	\Needspace{10\baselineskip}
	\section{Mon, Oct 09}
		\subsection{Meeting}
\lipsum[1-2]

		\subsection{Reading}
\lipsum[3-4]

	\Needspace{10\baselineskip}
	\section{Tue, Oct 10}
		\subsection{Meeting}
\lipsum[1-2]

		\subsection{Reading}
\lipsum[3-4]

	\Needspace{10\baselineskip}
	\section{Wed, Oct 11}
		\subsection{Meeting}
\lipsum[1-2]

		\subsection{Reading}
\lipsum[3-4]

	\Needspace{10\baselineskip}
	\section{Thu, Oct 12}
		\subsection{Meeting}
\lipsum[1-2]

		\subsection{Reading}
\lipsum[3-4]

	\Needspace{10\baselineskip}
	\section{Fri, Oct 13}
		\subsection{Meeting}
\lipsum[1-2]

		\subsection{Reading}
\lipsum[3-4]

	\Needspace{10\baselineskip}
	\section{Sat, Oct 14}
		\subsection{Meeting}
\lipsum[1-2]

		\subsection{Reading}
\lipsum[3-4]

	\Needspace{10\baselineskip}
	\section{Sun, Oct 15}
		\subsection{Meeting}
\lipsum[1-2]

		\subsection{Reading}
\lipsum[3-4]

	\Needspace{10\baselineskip}
	\section{Mon, Oct 16}
		\subsection{Meeting}
\lipsum[1-2]

		\subsection{Reading}
\lipsum[3-4]

	\Needspace{10\baselineskip}
	\section{Tue, Oct 17}
		\subsection{Meeting}
\lipsum[1-2]

		\subsection{Reading}
\lipsum[3-4]

	\Needspace{10\baselineskip}
	\section{Wed, Oct 18}
		\subsection{Meeting}
\lipsum[1-2]

		\subsection{Reading}
\lipsum[3-4]

	\Needspace{10\baselineskip}
	\section{Thu, Oct 19}
		\subsection{Meeting}
\lipsum[1-2]

		\subsection{Reading}
\lipsum[3-4]

	\Needspace{10\baselineskip}
	\section{Fri, Oct 20}
		\subsection{Meeting}
\lipsum[1-2]

		\subsection{Reading}
\lipsum[3-4]

	\Needspace{10\baselineskip}
	\section{Sat, Oct 21}
		\subsection{Meeting}
\lipsum[1-2]

		\subsection{Reading}
\lipsum[3-4]

	\Needspace{10\baselineskip}
	\section{Sun, Oct 22}
		\subsection{Meeting}
\lipsum[1-2]

		\subsection{Reading}
\lipsum[3-4]

	\Needspace{10\baselineskip}
	\section{Mon, Oct 23}
		\subsection{Meeting}
\lipsum[1-2]

		\subsection{Reading}
\lipsum[3-4]

	\Needspace{10\baselineskip}
	\section{Tue, Oct 24}
		\subsection{Meeting}
\lipsum[1-2]

		\subsection{Reading}
\lipsum[3-4]

	\Needspace{10\baselineskip}
	\section{Wed, Oct 25}
		\subsection{Meeting}
\lipsum[1-2]

		\subsection{Reading}
\lipsum[3-4]

	\Needspace{10\baselineskip}
	\section{Thu, Oct 26}
		\subsection{Meeting}
\lipsum[1-2]

		\subsection{Reading}
\lipsum[3-4]

	\Needspace{10\baselineskip}
	\section{Fri, Oct 27}
		\subsection{Meeting}
\lipsum[1-2]

		\subsection{Reading}
\lipsum[3-4]

	\Needspace{10\baselineskip}
	\section{Sat, Oct 28}
		\subsection{Meeting}
\lipsum[1-2]

		\subsection{Reading}
\lipsum[3-4]

	\Needspace{10\baselineskip}
	\section{Sun, Oct 29}
		\subsection{Meeting}
\lipsum[1-2]

		\subsection{Reading}
\lipsum[3-4]

	\Needspace{10\baselineskip}
	\section{Mon, Oct 30}
		\subsection{Meeting}
\lipsum[1-2]

		\subsection{Reading}
\lipsum[3-4]

	\Needspace{10\baselineskip}
	\section{Tue, Oct 31}
		\subsection{Meeting}
\lipsum[1-2]

		\subsection{Reading}
\lipsum[3-4]

}{}
% \IfFileExists{./Diary_2023/November/full_summary.tex}{\chapter{November}
	\section{Wed, Nov 01}
		\subsection{Meeting}
\lipsum[1-2]

		\subsection{Reading}
\lipsum[3-4]

	\section{Thu, Nov 02}
		\subsection{Meeting}
\lipsum[1-2]

		\subsection{Reading}
\lipsum[3-4]

	\section{Fri, Nov 03}
		\subsection{Meeting}
\lipsum[1-2]

		\subsection{Reading}
\lipsum[3-4]

	\section{Sat, Nov 04}
		\subsection{Meeting}
\lipsum[1-2]

		\subsection{Reading}
\lipsum[3-4]

	\section{Sun, Nov 05}
		\subsection{Meeting}
\lipsum[1-2]

		\subsection{Reading}
\lipsum[3-4]

	\section{Mon, Nov 06}
		\subsection{Meeting}
\lipsum[1-2]

		\subsection{Reading}
\lipsum[3-4]

	\section{Tue, Nov 07}
		\subsection{Meeting}
\lipsum[1-2]

		\subsection{Reading}
\lipsum[3-4]

	\section{Wed, Nov 08}
		\subsection{Meeting}
\lipsum[1-2]

		\subsection{Reading}
\lipsum[3-4]

	\section{Thu, Nov 09}
		\subsection{Meeting}
\lipsum[1-2]

		\subsection{Reading}
\lipsum[3-4]

	\section{Fri, Nov 10}
		\subsection{Meeting}
\lipsum[1-2]

		\subsection{Reading}
\lipsum[3-4]

	\section{Sat, Nov 11}
		\subsection{Meeting}
\lipsum[1-2]

		\subsection{Reading}
\lipsum[3-4]

	\section{Sun, Nov 12}
		\subsection{Meeting}
\lipsum[1-2]

		\subsection{Reading}
\lipsum[3-4]

	\section{Mon, Nov 13}
		\subsection{Meeting}
\lipsum[1-2]

		\subsection{Reading}
\lipsum[3-4]

	\section{Tue, Nov 14}
		\subsection{Meeting}
\lipsum[1-2]

		\subsection{Reading}
\lipsum[3-4]

	\section{Wed, Nov 15}
		\subsection{Meeting}
\lipsum[1-2]

		\subsection{Reading}
\lipsum[3-4]

	\section{Thu, Nov 16}
		\subsection{Meeting}
\lipsum[1-2]

		\subsection{Reading}
\lipsum[3-4]

	\section{Fri, Nov 17}
		\subsection{Meeting}
\lipsum[1-2]

		\subsection{Reading}
\lipsum[3-4]

	\section{Sat, Nov 18}
		\subsection{Meeting}
\lipsum[1-2]

		\subsection{Reading}
\lipsum[3-4]

	\section{Sun, Nov 19}
		\subsection{Meeting}
\lipsum[1-2]

		\subsection{Reading}
\lipsum[3-4]

	\section{Mon, Nov 20}
		\subsection{Meeting}
\lipsum[1-2]

		\subsection{Reading}
\lipsum[3-4]

	\section{Tue, Nov 21}
		\subsection{Meeting}
\lipsum[1-2]

		\subsection{Reading}
\lipsum[3-4]

	\section{Wed, Nov 22}
		\subsection{Meeting}
\lipsum[1-2]

		\subsection{Reading}
\lipsum[3-4]

	\section{Thu, Nov 23}
		\subsection{Meeting}
\lipsum[1-2]

		\subsection{Reading}
\lipsum[3-4]

	\section{Fri, Nov 24}
		\subsection{Meeting}
\lipsum[1-2]

		\subsection{Reading}
\lipsum[3-4]

	\section{Sat, Nov 25}
		\subsection{Meeting}
\lipsum[1-2]

		\subsection{Reading}
\lipsum[3-4]

	\section{Sun, Nov 26}
		\subsection{Meeting}
\lipsum[1-2]

		\subsection{Reading}
\lipsum[3-4]

	\section{Mon, Nov 27}
		\subsection{Meeting}
\lipsum[1-2]

		\subsection{Reading}
\lipsum[3-4]

	\section{Tue, Nov 28}
		\subsection{Meeting}
\lipsum[1-2]

		\subsection{Reading}
\lipsum[3-4]

	\section{Wed, Nov 29}
		\subsection{Meeting}
\lipsum[1-2]

		\subsection{Reading}
\lipsum[3-4]

	\section{Thu, Nov 30}
		\subsection{Meeting}
\lipsum[1-2]

		\subsection{Reading}
\lipsum[3-4]

}{}
% \IfFileExists{./Diary_2023/December/full_summary.tex}{\chapter{December}
	\section{Fri, Dec 01}
		\subsection{Meeting}
\lipsum[1-2]

		\subsection{Reading}
\lipsum[3-4]

	\section{Sat, Dec 02}
		\subsection{Meeting}
\lipsum[1-2]

		\subsection{Reading}
\lipsum[3-4]

	\section{Sun, Dec 03}
		\subsection{Meeting}
\lipsum[1-2]

		\subsection{Reading}
\lipsum[3-4]

	\section{Mon, Dec 04}
		\subsection{Meeting}
\lipsum[1-2]

		\subsection{Reading}
\lipsum[3-4]

	\section{Tue, Dec 05}
		\subsection{Meeting}
\lipsum[1-2]

		\subsection{Reading}
\lipsum[3-4]

	\section{Wed, Dec 06}
		\subsection{Meeting}
\lipsum[1-2]

		\subsection{Reading}
\lipsum[3-4]

	\section{Thu, Dec 07}
		\subsection{Meeting}
\lipsum[1-2]

		\subsection{Reading}
\lipsum[3-4]

	\section{Fri, Dec 08}
		\subsection{Meeting}
\lipsum[1-2]

		\subsection{Reading}
\lipsum[3-4]

	\section{Sat, Dec 09}
		\subsection{Meeting}
\lipsum[1-2]

		\subsection{Reading}
\lipsum[3-4]

	\section{Sun, Dec 10}
		\subsection{Meeting}
\lipsum[1-2]

		\subsection{Reading}
\lipsum[3-4]

	\section{Mon, Dec 11}
		\subsection{Meeting}
\lipsum[1-2]

		\subsection{Reading}
\lipsum[3-4]

	\section{Tue, Dec 12}
		\subsection{Meeting}
\lipsum[1-2]

		\subsection{Reading}
\lipsum[3-4]

	\section{Wed, Dec 13}
		\subsection{Meeting}
\lipsum[1-2]

		\subsection{Reading}
\lipsum[3-4]

	\section{Thu, Dec 14}
		\subsection{Meeting}
\lipsum[1-2]

		\subsection{Reading}
\lipsum[3-4]

	\section{Fri, Dec 15}
		\subsection{Meeting}
\lipsum[1-2]

		\subsection{Reading}
\lipsum[3-4]

	\section{Sat, Dec 16}
		\subsection{Meeting}
\lipsum[1-2]

		\subsection{Reading}
\lipsum[3-4]

	\section{Sun, Dec 17}
		\subsection{Meeting}
\lipsum[1-2]

		\subsection{Reading}
\lipsum[3-4]

	\section{Mon, Dec 18}
		\subsection{Meeting}
\lipsum[1-2]

		\subsection{Reading}
\lipsum[3-4]

	\section{Tue, Dec 19}
		\subsection{Meeting}
\lipsum[1-2]

		\subsection{Reading}
\lipsum[3-4]

	\section{Wed, Dec 20}
		\subsection{Meeting}
\lipsum[1-2]

		\subsection{Reading}
\lipsum[3-4]

	\section{Thu, Dec 21}
		\subsection{Meeting}
\lipsum[1-2]

		\subsection{Reading}
\lipsum[3-4]

	\section{Fri, Dec 22}
		\subsection{Meeting}
\lipsum[1-2]

		\subsection{Reading}
\lipsum[3-4]

	\section{Sat, Dec 23}
		\subsection{Meeting}
\lipsum[1-2]

		\subsection{Reading}
\lipsum[3-4]

	\section{Sun, Dec 24}
		\subsection{Meeting}
\lipsum[1-2]

		\subsection{Reading}
\lipsum[3-4]

	\section{Mon, Dec 25}
		\subsection{Meeting}
\lipsum[1-2]

		\subsection{Reading}
\lipsum[3-4]

	\section{Tue, Dec 26}
		\subsection{Meeting}
\lipsum[1-2]

		\subsection{Reading}
\lipsum[3-4]

	\section{Wed, Dec 27}
		\subsection{Meeting}
\lipsum[1-2]

		\subsection{Reading}
\lipsum[3-4]

	\section{Thu, Dec 28}
		\subsection{Meeting}
\lipsum[1-2]

		\subsection{Reading}
\lipsum[3-4]

	\section{Fri, Dec 29}
		\subsection{Meeting}
\lipsum[1-2]

		\subsection{Reading}
\lipsum[3-4]

	\section{Sat, Dec 30}
		\subsection{Meeting}
\lipsum[1-2]

		\subsection{Reading}
\lipsum[3-4]

	\section{Sun, Dec 31}
		\subsection{Meeting}
\lipsum[1-2]

		\subsection{Reading}
\lipsum[3-4]

}{}
% \IfFileExists{./Diary_2023/January/work_summary.tex}{\chapter{January}
	\Needspace{10\baselineskip}
	\section{Sun, Jan 01}
		\subsection{Meeting}
\lipsum[1-2]

	\Needspace{10\baselineskip}
	\section{Mon, Jan 02}
		\subsection{Meeting}
\lipsum[1-2]

	\Needspace{10\baselineskip}
	\section{Tue, Jan 03}
		\subsection{Meeting}
\lipsum[1-2]

	\Needspace{10\baselineskip}
	\section{Wed, Jan 04}
		\subsection{Meeting}
\lipsum[1-2]

	\Needspace{10\baselineskip}
	\section{Thu, Jan 05}
		\subsection{Meeting}
\lipsum[1-2]

	\Needspace{10\baselineskip}
	\section{Fri, Jan 06}
		\subsection{Meeting}
\lipsum[1-2]

	\Needspace{10\baselineskip}
	\section{Sat, Jan 07}
		\subsection{Meeting}
\lipsum[1-2]

	\Needspace{10\baselineskip}
	\section{Sun, Jan 08}
		\subsection{Meeting}
\lipsum[1-2]

	\Needspace{10\baselineskip}
	\section{Mon, Jan 09}
		\subsection{Meeting}
\lipsum[1-2]

	\Needspace{10\baselineskip}
	\section{Tue, Jan 10}
		\subsection{Meeting}
\lipsum[1-2]

	\Needspace{10\baselineskip}
	\section{Wed, Jan 11}
		\subsection{Meeting}
\lipsum[1-2]

	\Needspace{10\baselineskip}
	\section{Thu, Jan 12}
		\subsection{Meeting}
\lipsum[1-2]

	\Needspace{10\baselineskip}
	\section{Fri, Jan 13}
		\subsection{Meeting}
\lipsum[1-2]

	\Needspace{10\baselineskip}
	\section{Sat, Jan 14}
		\subsection{Meeting}
\lipsum[1-2]

	\Needspace{10\baselineskip}
	\section{Sun, Jan 15}
		\subsection{Meeting}
\lipsum[1-2]

	\Needspace{10\baselineskip}
	\section{Mon, Jan 16}
		\subsection{Meeting}
\lipsum[1-2]

	\Needspace{10\baselineskip}
	\section{Tue, Jan 17}
		\subsection{Meeting}
\lipsum[1-2]

	\Needspace{10\baselineskip}
	\section{Wed, Jan 18}
		\subsection{Meeting}
\lipsum[1-2]

	\Needspace{10\baselineskip}
	\section{Thu, Jan 19}
		\subsection{Meeting}
\lipsum[1-2]

	\Needspace{10\baselineskip}
	\section{Fri, Jan 20}
		\subsection{Meeting}
\lipsum[1-2]

	\Needspace{10\baselineskip}
	\section{Sat, Jan 21}
		\subsection{Meeting}
\lipsum[1-2]

	\Needspace{10\baselineskip}
	\section{Sun, Jan 22}
		\subsection{Meeting}
\lipsum[1-2]

	\Needspace{10\baselineskip}
	\section{Mon, Jan 23}
		\subsection{Meeting}
\lipsum[1-2]

	\Needspace{10\baselineskip}
	\section{Tue, Jan 24}
		\subsection{Meeting}
\lipsum[1-2]

	\Needspace{10\baselineskip}
	\section{Wed, Jan 25}
		\subsection{Meeting}
\lipsum[1-2]

	\Needspace{10\baselineskip}
	\section{Thu, Jan 26}
		\subsection{Meeting}
\lipsum[1-2]

	\Needspace{10\baselineskip}
	\section{Fri, Jan 27}
		\subsection{Meeting}
\lipsum[1-2]

	\Needspace{10\baselineskip}
	\section{Sat, Jan 28}
		\subsection{Meeting}
\lipsum[1-2]

	\Needspace{10\baselineskip}
	\section{Sun, Jan 29}
		\subsection{Meeting}
\lipsum[1-2]

	\Needspace{10\baselineskip}
	\section{Mon, Jan 30}
		\subsection{Meeting}
\lipsum[1-2]

	\Needspace{10\baselineskip}
	\section{Tue, Jan 31}
		\subsection{Meeting}
\lipsum[1-2]

}{}
\IfFileExists{./Diary_2023/February/work_summary.tex}{\chapter{February}
	\section{Wed, Feb 01}
		\subsection{Meeting}
\lipsum[1-2]

	\section{Thu, Feb 02}
		\subsection{Meeting}
\lipsum[1-2]

	\section{Fri, Feb 03}
		\subsection{Meeting}
\lipsum[1-2]

	\section{Sat, Feb 04}
		\subsection{Meeting}
\lipsum[1-2]

	\section{Sun, Feb 05}
		\subsection{Meeting}
\lipsum[1-2]

	\section{Mon, Feb 06}
		\subsection{Meeting}
\lipsum[1-2]

	\section{Tue, Feb 07}
		\subsection{Meeting}
\lipsum[1-2]

	\section{Wed, Feb 08}
		\subsection{Meeting}
\lipsum[1-2]

	\section{Thu, Feb 09}
		\subsection{Meeting}
\lipsum[1-2]

	\section{Fri, Feb 10}
		\subsection{Meeting}
\lipsum[1-2]

	\section{Sat, Feb 11}
		\subsection{Meeting}
\lipsum[1-2]

	\section{Sun, Feb 12}
		\subsection{Meeting}
\lipsum[1-2]

	\section{Mon, Feb 13}
		\subsection{Meeting}
\lipsum[1-2]

	\section{Tue, Feb 14}
		\subsection{Meeting}
\lipsum[1-2]

	\section{Wed, Feb 15}
		\subsection{Meeting}
\lipsum[1-2]

	\section{Thu, Feb 16}
		\subsection{Meeting}
\lipsum[1-2]

	\section{Fri, Feb 17}
		\subsection{Meeting}
\lipsum[1-2]

	\section{Sat, Feb 18}
		\subsection{Meeting}
\lipsum[1-2]

	\section{Sun, Feb 19}
		\subsection{Meeting}
\lipsum[1-2]

	\section{Mon, Feb 20}
		\subsection{Meeting}
\lipsum[1-2]

	\section{Tue, Feb 21}
		\subsection{Meeting}
\lipsum[1-2]

	\section{Wed, Feb 22}
		\subsection{Meeting}
\lipsum[1-2]

	\section{Thu, Feb 23}
		\subsection{Meeting}
\lipsum[1-2]

	\section{Fri, Feb 24}
		\subsection{Meeting}
\lipsum[1-2]

	\section{Sat, Feb 25}
		\subsection{Meeting}
\lipsum[1-2]

	\section{Sun, Feb 26}
		\subsection{Meeting}
\lipsum[1-2]

	\section{Mon, Feb 27}
		\subsection{Meeting}
\lipsum[1-2]

	\section{Tue, Feb 28}
		\subsection{Meeting}
\lipsum[1-2]

}{}
\IfFileExists{./Diary_2023/March/work_summary.tex}{\chapter{March}
	\Needspace{10\baselineskip}
	\section{Wed, Mar 01}
		\subsection{Meeting}
\lipsum[1-2]

	\Needspace{10\baselineskip}
	\section{Thu, Mar 02}
		\subsection{Meeting}
\lipsum[1-2]

	\Needspace{10\baselineskip}
	\section{Fri, Mar 03}
		\subsection{Meeting}
\lipsum[1-2]

	\Needspace{10\baselineskip}
	\section{Sat, Mar 04}
		\subsection{Meeting}
\lipsum[1-2]

	\Needspace{10\baselineskip}
	\section{Sun, Mar 05}
		\subsection{Meeting}
\lipsum[1-2]

	\Needspace{10\baselineskip}
	\section{Mon, Mar 06}
		\subsection{Meeting}
\lipsum[1-2]

	\Needspace{10\baselineskip}
	\section{Tue, Mar 07}
		\subsection{Meeting}
\lipsum[1-2]

	\Needspace{10\baselineskip}
	\section{Wed, Mar 08}
		\subsection{Meeting}
\lipsum[1-2]

	\Needspace{10\baselineskip}
	\section{Thu, Mar 09}
		\subsection{Meeting}
\lipsum[1-2]

	\Needspace{10\baselineskip}
	\section{Fri, Mar 10}
		\subsection{Meeting}
\lipsum[1-2]

	\Needspace{10\baselineskip}
	\section{Sat, Mar 11}
		\subsection{Meeting}
\lipsum[1-2]

	\Needspace{10\baselineskip}
	\section{Sun, Mar 12}
		\subsection{Meeting}
\lipsum[1-2]

	\Needspace{10\baselineskip}
	\section{Mon, Mar 13}
		\subsection{Meeting}
\lipsum[1-2]

	\Needspace{10\baselineskip}
	\section{Tue, Mar 14}
		\subsection{Meeting}
\lipsum[1-2]

	\Needspace{10\baselineskip}
	\section{Wed, Mar 15}
		\subsection{Meeting}
\lipsum[1-2]

	\Needspace{10\baselineskip}
	\section{Thu, Mar 16}
		\subsection{Meeting}
\lipsum[1-2]

	\Needspace{10\baselineskip}
	\section{Fri, Mar 17}
		\subsection{Meeting}
\lipsum[1-2]

	\Needspace{10\baselineskip}
	\section{Sat, Mar 18}
		\subsection{Meeting}
\lipsum[1-2]

	\Needspace{10\baselineskip}
	\section{Sun, Mar 19}
		\subsection{Meeting}
\lipsum[1-2]

	\Needspace{10\baselineskip}
	\section{Mon, Mar 20}
		\subsection{Meeting}
\lipsum[1-2]

	\Needspace{10\baselineskip}
	\section{Tue, Mar 21}
		\subsection{Meeting}
\lipsum[1-2]

	\Needspace{10\baselineskip}
	\section{Wed, Mar 22}
		\subsection{Meeting}
\lipsum[1-2]

	\Needspace{10\baselineskip}
	\section{Thu, Mar 23}
		\subsection{Meeting}
\lipsum[1-2]

	\Needspace{10\baselineskip}
	\section{Fri, Mar 24}
		\subsection{Meeting}
\lipsum[1-2]

	\Needspace{10\baselineskip}
	\section{Sat, Mar 25}
		\subsection{Meeting}
\lipsum[1-2]

	\Needspace{10\baselineskip}
	\section{Sun, Mar 26}
		\subsection{Meeting}
\lipsum[1-2]

	\Needspace{10\baselineskip}
	\section{Mon, Mar 27}
		\subsection{Meeting}
\lipsum[1-2]

	\Needspace{10\baselineskip}
	\section{Tue, Mar 28}
		\subsection{Meeting}
\lipsum[1-2]

	\Needspace{10\baselineskip}
	\section{Wed, Mar 29}
		\subsection{Meeting}
\lipsum[1-2]

	\Needspace{10\baselineskip}
	\section{Thu, Mar 30}
		\subsection{Meeting}
\lipsum[1-2]

	\Needspace{10\baselineskip}
	\section{Fri, Mar 31}
		\subsection{Meeting}
\lipsum[1-2]

}{}
\IfFileExists{./Diary_2023/April/work_summary.tex}{\chapter{April}
	\Needspace{10\baselineskip}
	\section{Sat, Apr 01}
		\subsection{Meeting}
\lipsum[1-2]

	\Needspace{10\baselineskip}
	\section{Sun, Apr 02}
		\subsection{Meeting}
\lipsum[1-2]

	\Needspace{10\baselineskip}
	\section{Mon, Apr 03}
		\subsection{Meeting}
\lipsum[1-2]

	\Needspace{10\baselineskip}
	\section{Tue, Apr 04}
		\subsection{Meeting}
\lipsum[1-2]

	\Needspace{10\baselineskip}
	\section{Wed, Apr 05}
		\subsection{Meeting}
\lipsum[1-2]

	\Needspace{10\baselineskip}
	\section{Thu, Apr 06}
		\subsection{Meeting}
\lipsum[1-2]

	\Needspace{10\baselineskip}
	\section{Fri, Apr 07}
		\subsection{Meeting}
\lipsum[1-2]

	\Needspace{10\baselineskip}
	\section{Sat, Apr 08}
		\subsection{Meeting}
\lipsum[1-2]

	\Needspace{10\baselineskip}
	\section{Sun, Apr 09}
		\subsection{Meeting}
\lipsum[1-2]

	\Needspace{10\baselineskip}
	\section{Mon, Apr 10}
		\subsection{Meeting}
\lipsum[1-2]

	\Needspace{10\baselineskip}
	\section{Tue, Apr 11}
		\subsection{Meeting}
\lipsum[1-2]

	\Needspace{10\baselineskip}
	\section{Wed, Apr 12}
		\subsection{Meeting}
\lipsum[1-2]

	\Needspace{10\baselineskip}
	\section{Thu, Apr 13}
		\subsection{Meeting}
\lipsum[1-2]

	\Needspace{10\baselineskip}
	\section{Fri, Apr 14}
		\subsection{Meeting}
\lipsum[1-2]

	\Needspace{10\baselineskip}
	\section{Sat, Apr 15}
		\subsection{Meeting}
\lipsum[1-2]

	\Needspace{10\baselineskip}
	\section{Sun, Apr 16}
		\subsection{Meeting}
\lipsum[1-2]

	\Needspace{10\baselineskip}
	\section{Mon, Apr 17}
		\subsection{Meeting}
\lipsum[1-2]

	\Needspace{10\baselineskip}
	\section{Tue, Apr 18}
		\subsection{Meeting}
\lipsum[1-2]

	\Needspace{10\baselineskip}
	\section{Wed, Apr 19}
		\subsection{Meeting}
\lipsum[1-2]

	\Needspace{10\baselineskip}
	\section{Thu, Apr 20}
		\subsection{Meeting}
\lipsum[1-2]

	\Needspace{10\baselineskip}
	\section{Fri, Apr 21}
		\subsection{Meeting}
\lipsum[1-2]

	\Needspace{10\baselineskip}
	\section{Sat, Apr 22}
		\subsection{Meeting}
\lipsum[1-2]

	\Needspace{10\baselineskip}
	\section{Sun, Apr 23}
		\subsection{Meeting}
\lipsum[1-2]

	\Needspace{10\baselineskip}
	\section{Mon, Apr 24}
		\subsection{Meeting}
\lipsum[1-2]

	\Needspace{10\baselineskip}
	\section{Tue, Apr 25}
		\subsection{Meeting}
\lipsum[1-2]

	\Needspace{10\baselineskip}
	\section{Wed, Apr 26}
		\subsection{Meeting}
\lipsum[1-2]

	\Needspace{10\baselineskip}
	\section{Thu, Apr 27}
		\subsection{Meeting}
\lipsum[1-2]

	\Needspace{10\baselineskip}
	\section{Fri, Apr 28}
		\subsection{Meeting}
\lipsum[1-2]

	\Needspace{10\baselineskip}
	\section{Sat, Apr 29}
		\subsection{Meeting}
\lipsum[1-2]

	\Needspace{10\baselineskip}
	\section{Sun, Apr 30}
		\subsection{Meeting}
\lipsum[1-2]

}{}
\IfFileExists{./Diary_2023/May/work_summary.tex}{\chapter{May}
	\Needspace{10\baselineskip}
	\section{Mon, May 01}
		\subsection{Meeting}
\lipsum[1-2]

	\Needspace{10\baselineskip}
	\section{Tue, May 02}
		\subsection{Meeting}
\lipsum[1-2]

	\Needspace{10\baselineskip}
	\section{Wed, May 03}
		\subsection{Meeting}
\lipsum[1-2]

	\Needspace{10\baselineskip}
	\section{Thu, May 04}
		\subsection{Meeting}
\lipsum[1-2]

	\Needspace{10\baselineskip}
	\section{Fri, May 05}
		\subsection{Meeting}
\lipsum[1-2]

	\Needspace{10\baselineskip}
	\section{Sat, May 06}
		\subsection{Meeting}
\lipsum[1-2]

	\Needspace{10\baselineskip}
	\section{Sun, May 07}
		\subsection{Meeting}
\lipsum[1-2]

	\Needspace{10\baselineskip}
	\section{Mon, May 08}
		\subsection{Meeting}
\lipsum[1-2]

	\Needspace{10\baselineskip}
	\section{Tue, May 09}
		\subsection{Meeting}
\lipsum[1-2]

	\Needspace{10\baselineskip}
	\section{Wed, May 10}
		\subsection{Meeting}
\lipsum[1-2]

	\Needspace{10\baselineskip}
	\section{Thu, May 11}
		\subsection{Meeting}
\lipsum[1-2]

	\Needspace{10\baselineskip}
	\section{Fri, May 12}
		\subsection{Meeting}
\lipsum[1-2]

	\Needspace{10\baselineskip}
	\section{Sat, May 13}
		\subsection{Meeting}
\lipsum[1-2]

	\Needspace{10\baselineskip}
	\section{Sun, May 14}
		\subsection{Meeting}
\lipsum[1-2]

	\Needspace{10\baselineskip}
	\section{Mon, May 15}
		\subsection{Meeting}
\lipsum[1-2]

	\Needspace{10\baselineskip}
	\section{Tue, May 16}
		\subsection{Meeting}
\lipsum[1-2]

	\Needspace{10\baselineskip}
	\section{Wed, May 17}
		\subsection{Meeting}
\lipsum[1-2]

	\Needspace{10\baselineskip}
	\section{Thu, May 18}
		\subsection{Meeting}
\lipsum[1-2]

	\Needspace{10\baselineskip}
	\section{Fri, May 19}
		\subsection{Meeting}
\lipsum[1-2]

	\Needspace{10\baselineskip}
	\section{Sat, May 20}
		\subsection{Meeting}
\lipsum[1-2]

	\Needspace{10\baselineskip}
	\section{Sun, May 21}
		\subsection{Meeting}
\lipsum[1-2]

	\Needspace{10\baselineskip}
	\section{Mon, May 22}
		\subsection{Meeting}
\lipsum[1-2]

	\Needspace{10\baselineskip}
	\section{Tue, May 23}
		\subsection{Meeting}
\lipsum[1-2]

	\Needspace{10\baselineskip}
	\section{Wed, May 24}
		\subsection{Meeting}
\lipsum[1-2]

	\Needspace{10\baselineskip}
	\section{Thu, May 25}
		\subsection{Meeting}
\lipsum[1-2]

	\Needspace{10\baselineskip}
	\section{Fri, May 26}
		\subsection{Meeting}
\lipsum[1-2]

	\Needspace{10\baselineskip}
	\section{Sat, May 27}
		\subsection{Meeting}
\lipsum[1-2]

	\Needspace{10\baselineskip}
	\section{Sun, May 28}
		\subsection{Meeting}
\lipsum[1-2]

	\Needspace{10\baselineskip}
	\section{Mon, May 29}
		\subsection{Meeting}
\lipsum[1-2]

	\Needspace{10\baselineskip}
	\section{Tue, May 30}
		\subsection{Meeting}
\lipsum[1-2]

	\Needspace{10\baselineskip}
	\section{Wed, May 31}
		\subsection{Meeting}
\lipsum[1-2]

}{}
\IfFileExists{./Diary_2023/June/work_summary.tex}{\chapter{June}
	\section{Thu, Jun 01}
		\subsection{Meeting}
\lipsum[1-2]

	\section{Fri, Jun 02}
		\subsection{Meeting}
\lipsum[1-2]

	\section{Sat, Jun 03}
		\subsection{Meeting}
\lipsum[1-2]

	\section{Sun, Jun 04}
		\subsection{Meeting}
\lipsum[1-2]

	\section{Mon, Jun 05}
		\subsection{Meeting}
\lipsum[1-2]

	\section{Tue, Jun 06}
		\subsection{Meeting}
\lipsum[1-2]

	\section{Wed, Jun 07}
		\subsection{Meeting}
\lipsum[1-2]

	\section{Thu, Jun 08}
		\subsection{Meeting}
\lipsum[1-2]

	\section{Fri, Jun 09}
		\subsection{Meeting}
\lipsum[1-2]

	\section{Sat, Jun 10}
		\subsection{Meeting}
\lipsum[1-2]

	\section{Sun, Jun 11}
		\subsection{Meeting}
\lipsum[1-2]

	\section{Mon, Jun 12}
		\subsection{Meeting}
\lipsum[1-2]

	\section{Tue, Jun 13}
		\subsection{Meeting}
\lipsum[1-2]

	\section{Wed, Jun 14}
		\subsection{Meeting}
\lipsum[1-2]

	\section{Thu, Jun 15}
		\subsection{Meeting}
\lipsum[1-2]

	\section{Fri, Jun 16}
		\subsection{Meeting}
\lipsum[1-2]

	\section{Sat, Jun 17}
		\subsection{Meeting}
\lipsum[1-2]

	\section{Sun, Jun 18}
		\subsection{Meeting}
\lipsum[1-2]

	\section{Mon, Jun 19}
		\subsection{Meeting}
\lipsum[1-2]

	\section{Tue, Jun 20}
		\subsection{Meeting}
\lipsum[1-2]

	\section{Wed, Jun 21}
		\subsection{Meeting}
\lipsum[1-2]

	\section{Thu, Jun 22}
		\subsection{Meeting}
\lipsum[1-2]

	\section{Fri, Jun 23}
		\subsection{Meeting}
\lipsum[1-2]

	\section{Sat, Jun 24}
		\subsection{Meeting}
\lipsum[1-2]

	\section{Sun, Jun 25}
		\subsection{Meeting}
\lipsum[1-2]

	\section{Mon, Jun 26}
		\subsection{Meeting}
\lipsum[1-2]

	\section{Tue, Jun 27}
		\subsection{Meeting}
\lipsum[1-2]

	\section{Wed, Jun 28}
		\subsection{Meeting}
\lipsum[1-2]

	\section{Thu, Jun 29}
		\subsection{Meeting}
\lipsum[1-2]

	\section{Fri, Jun 30}
		\subsection{Meeting}
\lipsum[1-2]

}{}
\IfFileExists{./Diary_2023/July/work_summary.tex}{\chapter{July}
	\Needspace{10\baselineskip}
	\section{Sat, Jul 01}
		\subsection{Meeting}
\lipsum[1-2]

	\Needspace{10\baselineskip}
	\section{Sun, Jul 02}
		\subsection{Meeting}
\lipsum[1-2]

	\Needspace{10\baselineskip}
	\section{Mon, Jul 03}
		\subsection{Meeting}
\lipsum[1-2]

	\Needspace{10\baselineskip}
	\section{Tue, Jul 04}
		\subsection{Meeting}
\lipsum[1-2]

	\Needspace{10\baselineskip}
	\section{Wed, Jul 05}
		\subsection{Meeting}
\lipsum[1-2]

	\Needspace{10\baselineskip}
	\section{Thu, Jul 06}
		\subsection{Meeting}
\lipsum[1-2]

	\Needspace{10\baselineskip}
	\section{Fri, Jul 07}
		\subsection{Meeting}
\lipsum[1-2]

	\Needspace{10\baselineskip}
	\section{Sat, Jul 08}
		\subsection{Meeting}
\lipsum[1-2]

	\Needspace{10\baselineskip}
	\section{Sun, Jul 09}
		\subsection{Meeting}
\lipsum[1-2]

	\Needspace{10\baselineskip}
	\section{Mon, Jul 10}
		\subsection{Meeting}
\lipsum[1-2]

	\Needspace{10\baselineskip}
	\section{Tue, Jul 11}
		\subsection{Meeting}
\lipsum[1-2]

	\Needspace{10\baselineskip}
	\section{Wed, Jul 12}
		\subsection{Meeting}
\lipsum[1-2]

	\Needspace{10\baselineskip}
	\section{Thu, Jul 13}
		\subsection{Meeting}
\lipsum[1-2]

	\Needspace{10\baselineskip}
	\section{Fri, Jul 14}
		\subsection{Meeting}
\lipsum[1-2]

	\Needspace{10\baselineskip}
	\section{Sat, Jul 15}
		\subsection{Meeting}
\lipsum[1-2]

	\Needspace{10\baselineskip}
	\section{Sun, Jul 16}
		\subsection{Meeting}
\lipsum[1-2]

	\Needspace{10\baselineskip}
	\section{Mon, Jul 17}
		\subsection{Meeting}
\lipsum[1-2]

	\Needspace{10\baselineskip}
	\section{Tue, Jul 18}
		\subsection{Meeting}
\lipsum[1-2]

	\Needspace{10\baselineskip}
	\section{Wed, Jul 19}
		\subsection{Meeting}
\lipsum[1-2]

	\Needspace{10\baselineskip}
	\section{Thu, Jul 20}
		\subsection{Meeting}
\lipsum[1-2]

	\Needspace{10\baselineskip}
	\section{Fri, Jul 21}
		\subsection{Meeting}
\lipsum[1-2]

	\Needspace{10\baselineskip}
	\section{Sat, Jul 22}
		\subsection{Meeting}
\lipsum[1-2]

	\Needspace{10\baselineskip}
	\section{Sun, Jul 23}
		\subsection{Meeting}
\lipsum[1-2]

	\Needspace{10\baselineskip}
	\section{Mon, Jul 24}
		\subsection{Meeting}
\lipsum[1-2]

	\Needspace{10\baselineskip}
	\section{Tue, Jul 25}
		\subsection{Meeting}
\lipsum[1-2]

	\Needspace{10\baselineskip}
	\section{Wed, Jul 26}
		\subsection{Meeting}
\lipsum[1-2]

	\Needspace{10\baselineskip}
	\section{Thu, Jul 27}
		\subsection{Meeting}
\lipsum[1-2]

	\Needspace{10\baselineskip}
	\section{Fri, Jul 28}
		\subsection{Meeting}
\lipsum[1-2]

	\Needspace{10\baselineskip}
	\section{Sat, Jul 29}
		\subsection{Meeting}
\lipsum[1-2]

	\Needspace{10\baselineskip}
	\section{Sun, Jul 30}
		\subsection{Meeting}
\lipsum[1-2]

	\Needspace{10\baselineskip}
	\section{Mon, Jul 31}
		\subsection{Meeting}
\lipsum[1-2]

}{}
\IfFileExists{./Diary_2023/August/work_summary.tex}{\chapter{August}
	\Needspace{10\baselineskip}
	\section{Tue, Aug 01}
		\subsection{Meeting}
\lipsum[1-2]

	\Needspace{10\baselineskip}
	\section{Wed, Aug 02}
		\subsection{Meeting}
\lipsum[1-2]

	\Needspace{10\baselineskip}
	\section{Thu, Aug 03}
		\subsection{Meeting}
\lipsum[1-2]

	\Needspace{10\baselineskip}
	\section{Fri, Aug 04}
		\subsection{Meeting}
\lipsum[1-2]

	\Needspace{10\baselineskip}
	\section{Sat, Aug 05}
		\subsection{Meeting}
\lipsum[1-2]

	\Needspace{10\baselineskip}
	\section{Sun, Aug 06}
		\subsection{Meeting}
\lipsum[1-2]

	\Needspace{10\baselineskip}
	\section{Mon, Aug 07}
		\subsection{Meeting}
\lipsum[1-2]

	\Needspace{10\baselineskip}
	\section{Tue, Aug 08}
		\subsection{Meeting}
\lipsum[1-2]

	\Needspace{10\baselineskip}
	\section{Wed, Aug 09}
		\subsection{Meeting}
\lipsum[1-2]

	\Needspace{10\baselineskip}
	\section{Thu, Aug 10}
		\subsection{Meeting}
\lipsum[1-2]

	\Needspace{10\baselineskip}
	\section{Fri, Aug 11}
		\subsection{Meeting}
\lipsum[1-2]

	\Needspace{10\baselineskip}
	\section{Sat, Aug 12}
		\subsection{Meeting}
\lipsum[1-2]

	\Needspace{10\baselineskip}
	\section{Sun, Aug 13}
		\subsection{Meeting}
\lipsum[1-2]

	\Needspace{10\baselineskip}
	\section{Mon, Aug 14}
		\subsection{Meeting}
\lipsum[1-2]

	\Needspace{10\baselineskip}
	\section{Tue, Aug 15}
		\subsection{Meeting}
\lipsum[1-2]

	\Needspace{10\baselineskip}
	\section{Wed, Aug 16}
		\subsection{Meeting}
\lipsum[1-2]

	\Needspace{10\baselineskip}
	\section{Thu, Aug 17}
		\subsection{Meeting}
\lipsum[1-2]

	\Needspace{10\baselineskip}
	\section{Fri, Aug 18}
		\subsection{Meeting}
\lipsum[1-2]

	\Needspace{10\baselineskip}
	\section{Sat, Aug 19}
		\subsection{Meeting}
\lipsum[1-2]

	\Needspace{10\baselineskip}
	\section{Sun, Aug 20}
		\subsection{Meeting}
\lipsum[1-2]

	\Needspace{10\baselineskip}
	\section{Mon, Aug 21}
		\subsection{Meeting}
\lipsum[1-2]

	\Needspace{10\baselineskip}
	\section{Tue, Aug 22}
		\subsection{Meeting}
\lipsum[1-2]

	\Needspace{10\baselineskip}
	\section{Wed, Aug 23}
		\subsection{Meeting}
\lipsum[1-2]

	\Needspace{10\baselineskip}
	\section{Thu, Aug 24}
		\subsection{Meeting}
\lipsum[1-2]

	\Needspace{10\baselineskip}
	\section{Fri, Aug 25}
		\subsection{Meeting}
\lipsum[1-2]

	\Needspace{10\baselineskip}
	\section{Sat, Aug 26}
		\subsection{Meeting}
\lipsum[1-2]

	\Needspace{10\baselineskip}
	\section{Sun, Aug 27}
		\subsection{Meeting}
\lipsum[1-2]

	\Needspace{10\baselineskip}
	\section{Mon, Aug 28}
		\subsection{Meeting}
\lipsum[1-2]

	\Needspace{10\baselineskip}
	\section{Tue, Aug 29}
		\subsection{Meeting}
\lipsum[1-2]

	\Needspace{10\baselineskip}
	\section{Wed, Aug 30}
		\subsection{Meeting}
\lipsum[1-2]

	\Needspace{10\baselineskip}
	\section{Thu, Aug 31}
		\subsection{Meeting}
\lipsum[1-2]

}{}
\IfFileExists{./Diary_2023/September/work_summary.tex}{\chapter{September}
	\section{Fri, Sep 01}
		\subsection{Meeting}
\lipsum[1-2]

	\section{Sat, Sep 02}
		\subsection{Meeting}
\lipsum[1-2]

	\section{Sun, Sep 03}
		\subsection{Meeting}
\lipsum[1-2]

	\section{Mon, Sep 04}
		\subsection{Meeting}
\lipsum[1-2]

	\section{Tue, Sep 05}
		\subsection{Meeting}
\lipsum[1-2]

	\section{Wed, Sep 06}
		\subsection{Meeting}
\lipsum[1-2]

	\section{Thu, Sep 07}
		\subsection{Meeting}
\lipsum[1-2]

	\section{Fri, Sep 08}
		\subsection{Meeting}
\lipsum[1-2]

	\section{Sat, Sep 09}
		\subsection{Meeting}
\lipsum[1-2]

	\section{Sun, Sep 10}
		\subsection{Meeting}
\lipsum[1-2]

	\section{Mon, Sep 11}
		\subsection{Meeting}
\lipsum[1-2]

	\section{Tue, Sep 12}
		\subsection{Meeting}
\lipsum[1-2]

	\section{Wed, Sep 13}
		\subsection{Meeting}
\lipsum[1-2]

	\section{Thu, Sep 14}
		\subsection{Meeting}
\lipsum[1-2]

	\section{Fri, Sep 15}
		\subsection{Meeting}
\lipsum[1-2]

	\section{Sat, Sep 16}
		\subsection{Meeting}
\lipsum[1-2]

	\section{Sun, Sep 17}
		\subsection{Meeting}
\lipsum[1-2]

	\section{Mon, Sep 18}
		\subsection{Meeting}
\lipsum[1-2]

	\section{Tue, Sep 19}
		\subsection{Meeting}
\lipsum[1-2]

	\section{Wed, Sep 20}
		\subsection{Meeting}
\lipsum[1-2]

	\section{Thu, Sep 21}
		\subsection{Meeting}
\lipsum[1-2]

	\section{Fri, Sep 22}
		\subsection{Meeting}
\lipsum[1-2]

	\section{Sat, Sep 23}
		\subsection{Meeting}
\lipsum[1-2]

	\section{Sun, Sep 24}
		\subsection{Meeting}
\lipsum[1-2]

	\section{Mon, Sep 25}
		\subsection{Meeting}
\lipsum[1-2]

	\section{Tue, Sep 26}
		\subsection{Meeting}
\lipsum[1-2]

	\section{Wed, Sep 27}
		\subsection{Meeting}
\lipsum[1-2]

	\section{Thu, Sep 28}
		\subsection{Meeting}
\lipsum[1-2]

	\section{Fri, Sep 29}
		\subsection{Meeting}
\lipsum[1-2]

	\section{Sat, Sep 30}
		\subsection{Meeting}
\lipsum[1-2]

}{}
\IfFileExists{./Diary_2023/October/work_summary.tex}{\chapter{October}
	\Needspace{10\baselineskip}
	\section{Sun, Oct 01}
		\subsection{Meeting}
\lipsum[1-2]

	\Needspace{10\baselineskip}
	\section{Mon, Oct 02}
		\subsection{Meeting}
\lipsum[1-2]

	\Needspace{10\baselineskip}
	\section{Tue, Oct 03}
		\subsection{Meeting}
\lipsum[1-2]

	\Needspace{10\baselineskip}
	\section{Wed, Oct 04}
		\subsection{Meeting}
\lipsum[1-2]

	\Needspace{10\baselineskip}
	\section{Thu, Oct 05}
		\subsection{Meeting}
\lipsum[1-2]

	\Needspace{10\baselineskip}
	\section{Fri, Oct 06}
		\subsection{Meeting}
\lipsum[1-2]

	\Needspace{10\baselineskip}
	\section{Sat, Oct 07}
		\subsection{Meeting}
\lipsum[1-2]

	\Needspace{10\baselineskip}
	\section{Sun, Oct 08}
		\subsection{Meeting}
\lipsum[1-2]

	\Needspace{10\baselineskip}
	\section{Mon, Oct 09}
		\subsection{Meeting}
\lipsum[1-2]

	\Needspace{10\baselineskip}
	\section{Tue, Oct 10}
		\subsection{Meeting}
\lipsum[1-2]

	\Needspace{10\baselineskip}
	\section{Wed, Oct 11}
		\subsection{Meeting}
\lipsum[1-2]

	\Needspace{10\baselineskip}
	\section{Thu, Oct 12}
		\subsection{Meeting}
\lipsum[1-2]

	\Needspace{10\baselineskip}
	\section{Fri, Oct 13}
		\subsection{Meeting}
\lipsum[1-2]

	\Needspace{10\baselineskip}
	\section{Sat, Oct 14}
		\subsection{Meeting}
\lipsum[1-2]

	\Needspace{10\baselineskip}
	\section{Sun, Oct 15}
		\subsection{Meeting}
\lipsum[1-2]

	\Needspace{10\baselineskip}
	\section{Mon, Oct 16}
		\subsection{Meeting}
\lipsum[1-2]

	\Needspace{10\baselineskip}
	\section{Tue, Oct 17}
		\subsection{Meeting}
\lipsum[1-2]

	\Needspace{10\baselineskip}
	\section{Wed, Oct 18}
		\subsection{Meeting}
\lipsum[1-2]

	\Needspace{10\baselineskip}
	\section{Thu, Oct 19}
		\subsection{Meeting}
\lipsum[1-2]

	\Needspace{10\baselineskip}
	\section{Fri, Oct 20}
		\subsection{Meeting}
\lipsum[1-2]

	\Needspace{10\baselineskip}
	\section{Sat, Oct 21}
		\subsection{Meeting}
\lipsum[1-2]

	\Needspace{10\baselineskip}
	\section{Sun, Oct 22}
		\subsection{Meeting}
\lipsum[1-2]

	\Needspace{10\baselineskip}
	\section{Mon, Oct 23}
		\subsection{Meeting}
\lipsum[1-2]

	\Needspace{10\baselineskip}
	\section{Tue, Oct 24}
		\subsection{Meeting}
\lipsum[1-2]

	\Needspace{10\baselineskip}
	\section{Wed, Oct 25}
		\subsection{Meeting}
\lipsum[1-2]

	\Needspace{10\baselineskip}
	\section{Thu, Oct 26}
		\subsection{Meeting}
\lipsum[1-2]

	\Needspace{10\baselineskip}
	\section{Fri, Oct 27}
		\subsection{Meeting}
\lipsum[1-2]

	\Needspace{10\baselineskip}
	\section{Sat, Oct 28}
		\subsection{Meeting}
\lipsum[1-2]

	\Needspace{10\baselineskip}
	\section{Sun, Oct 29}
		\subsection{Meeting}
\lipsum[1-2]

	\Needspace{10\baselineskip}
	\section{Mon, Oct 30}
		\subsection{Meeting}
\lipsum[1-2]

	\Needspace{10\baselineskip}
	\section{Tue, Oct 31}
		\subsection{Meeting}
\lipsum[1-2]

}{}
\IfFileExists{./Diary_2023/November/work_summary.tex}{\chapter{November}
	\section{Wed, Nov 01}
		\subsection{Meeting}
\lipsum[1-2]

	\section{Thu, Nov 02}
		\subsection{Meeting}
\lipsum[1-2]

	\section{Fri, Nov 03}
		\subsection{Meeting}
\lipsum[1-2]

	\section{Sat, Nov 04}
		\subsection{Meeting}
\lipsum[1-2]

	\section{Sun, Nov 05}
		\subsection{Meeting}
\lipsum[1-2]

	\section{Mon, Nov 06}
		\subsection{Meeting}
\lipsum[1-2]

	\section{Tue, Nov 07}
		\subsection{Meeting}
\lipsum[1-2]

	\section{Wed, Nov 08}
		\subsection{Meeting}
\lipsum[1-2]

	\section{Thu, Nov 09}
		\subsection{Meeting}
\lipsum[1-2]

	\section{Fri, Nov 10}
		\subsection{Meeting}
\lipsum[1-2]

	\section{Sat, Nov 11}
		\subsection{Meeting}
\lipsum[1-2]

	\section{Sun, Nov 12}
		\subsection{Meeting}
\lipsum[1-2]

	\section{Mon, Nov 13}
		\subsection{Meeting}
\lipsum[1-2]

	\section{Tue, Nov 14}
		\subsection{Meeting}
\lipsum[1-2]

	\section{Wed, Nov 15}
		\subsection{Meeting}
\lipsum[1-2]

	\section{Thu, Nov 16}
		\subsection{Meeting}
\lipsum[1-2]

	\section{Fri, Nov 17}
		\subsection{Meeting}
\lipsum[1-2]

	\section{Sat, Nov 18}
		\subsection{Meeting}
\lipsum[1-2]

	\section{Sun, Nov 19}
		\subsection{Meeting}
\lipsum[1-2]

	\section{Mon, Nov 20}
		\subsection{Meeting}
\lipsum[1-2]

	\section{Tue, Nov 21}
		\subsection{Meeting}
\lipsum[1-2]

	\section{Wed, Nov 22}
		\subsection{Meeting}
\lipsum[1-2]

	\section{Thu, Nov 23}
		\subsection{Meeting}
\lipsum[1-2]

	\section{Fri, Nov 24}
		\subsection{Meeting}
\lipsum[1-2]

	\section{Sat, Nov 25}
		\subsection{Meeting}
\lipsum[1-2]

	\section{Sun, Nov 26}
		\subsection{Meeting}
\lipsum[1-2]

	\section{Mon, Nov 27}
		\subsection{Meeting}
\lipsum[1-2]

	\section{Tue, Nov 28}
		\subsection{Meeting}
\lipsum[1-2]

	\section{Wed, Nov 29}
		\subsection{Meeting}
\lipsum[1-2]

	\section{Thu, Nov 30}
		\subsection{Meeting}
\lipsum[1-2]

}{}
\IfFileExists{./Diary_2023/December/work_summary.tex}{\chapter{December}
	\section{Fri, Dec 01}
		\subsection{Meeting}
\lipsum[1-2]

	\section{Sat, Dec 02}
		\subsection{Meeting}
\lipsum[1-2]

	\section{Sun, Dec 03}
		\subsection{Meeting}
\lipsum[1-2]

	\section{Mon, Dec 04}
		\subsection{Meeting}
\lipsum[1-2]

	\section{Tue, Dec 05}
		\subsection{Meeting}
\lipsum[1-2]

	\section{Wed, Dec 06}
		\subsection{Meeting}
\lipsum[1-2]

	\section{Thu, Dec 07}
		\subsection{Meeting}
\lipsum[1-2]

	\section{Fri, Dec 08}
		\subsection{Meeting}
\lipsum[1-2]

	\section{Sat, Dec 09}
		\subsection{Meeting}
\lipsum[1-2]

	\section{Sun, Dec 10}
		\subsection{Meeting}
\lipsum[1-2]

	\section{Mon, Dec 11}
		\subsection{Meeting}
\lipsum[1-2]

	\section{Tue, Dec 12}
		\subsection{Meeting}
\lipsum[1-2]

	\section{Wed, Dec 13}
		\subsection{Meeting}
\lipsum[1-2]

	\section{Thu, Dec 14}
		\subsection{Meeting}
\lipsum[1-2]

	\section{Fri, Dec 15}
		\subsection{Meeting}
\lipsum[1-2]

	\section{Sat, Dec 16}
		\subsection{Meeting}
\lipsum[1-2]

	\section{Sun, Dec 17}
		\subsection{Meeting}
\lipsum[1-2]

	\section{Mon, Dec 18}
		\subsection{Meeting}
\lipsum[1-2]

	\section{Tue, Dec 19}
		\subsection{Meeting}
\lipsum[1-2]

	\section{Wed, Dec 20}
		\subsection{Meeting}
\lipsum[1-2]

	\section{Thu, Dec 21}
		\subsection{Meeting}
\lipsum[1-2]

	\section{Fri, Dec 22}
		\subsection{Meeting}
\lipsum[1-2]

	\section{Sat, Dec 23}
		\subsection{Meeting}
\lipsum[1-2]

	\section{Sun, Dec 24}
		\subsection{Meeting}
\lipsum[1-2]

	\section{Mon, Dec 25}
		\subsection{Meeting}
\lipsum[1-2]

	\section{Tue, Dec 26}
		\subsection{Meeting}
\lipsum[1-2]

	\section{Wed, Dec 27}
		\subsection{Meeting}
\lipsum[1-2]

	\section{Thu, Dec 28}
		\subsection{Meeting}
\lipsum[1-2]

	\section{Fri, Dec 29}
		\subsection{Meeting}
\lipsum[1-2]

	\section{Sat, Dec 30}
		\subsection{Meeting}
\lipsum[1-2]

	\section{Sun, Dec 31}
		\subsection{Meeting}
\lipsum[1-2]

}{}
% \IfFileExists{./Diary_2023/January/leisure_summary.tex}{\chapter{January}
	\Needspace{10\baselineskip}
	\section{Sun, Jan 01}
		\subsection{Reading}
\lipsum[3-4]

	\Needspace{10\baselineskip}
	\section{Mon, Jan 02}
		\subsection{Reading}
\lipsum[3-4]

	\Needspace{10\baselineskip}
	\section{Tue, Jan 03}
		\subsection{Reading}
\lipsum[3-4]

	\Needspace{10\baselineskip}
	\section{Wed, Jan 04}
		\subsection{Reading}
\lipsum[3-4]

	\Needspace{10\baselineskip}
	\section{Thu, Jan 05}
		\subsection{Reading}
\lipsum[3-4]

	\Needspace{10\baselineskip}
	\section{Fri, Jan 06}
		\subsection{Reading}
\lipsum[3-4]

	\Needspace{10\baselineskip}
	\section{Sat, Jan 07}
		\subsection{Reading}
\lipsum[3-4]

	\Needspace{10\baselineskip}
	\section{Sun, Jan 08}
		\subsection{Reading}
\lipsum[3-4]

	\Needspace{10\baselineskip}
	\section{Mon, Jan 09}
		\subsection{Reading}
\lipsum[3-4]

	\Needspace{10\baselineskip}
	\section{Tue, Jan 10}
		\subsection{Reading}
\lipsum[3-4]

	\Needspace{10\baselineskip}
	\section{Wed, Jan 11}
		\subsection{Reading}
\lipsum[3-4]

	\Needspace{10\baselineskip}
	\section{Thu, Jan 12}
		\subsection{Reading}
\lipsum[3-4]

	\Needspace{10\baselineskip}
	\section{Fri, Jan 13}
		\subsection{Reading}
\lipsum[3-4]

	\Needspace{10\baselineskip}
	\section{Sat, Jan 14}
		\subsection{Reading}
\lipsum[3-4]

	\Needspace{10\baselineskip}
	\section{Sun, Jan 15}
		\subsection{Reading}
\lipsum[3-4]

	\Needspace{10\baselineskip}
	\section{Mon, Jan 16}
		\subsection{Reading}
\lipsum[3-4]

	\Needspace{10\baselineskip}
	\section{Tue, Jan 17}
		\subsection{Reading}
\lipsum[3-4]

	\Needspace{10\baselineskip}
	\section{Wed, Jan 18}
		\subsection{Reading}
\lipsum[3-4]

	\Needspace{10\baselineskip}
	\section{Thu, Jan 19}
		\subsection{Reading}
\lipsum[3-4]

	\Needspace{10\baselineskip}
	\section{Fri, Jan 20}
		\subsection{Reading}
\lipsum[3-4]

	\Needspace{10\baselineskip}
	\section{Sat, Jan 21}
		\subsection{Reading}
\lipsum[3-4]

	\Needspace{10\baselineskip}
	\section{Sun, Jan 22}
		\subsection{Reading}
\lipsum[3-4]

	\Needspace{10\baselineskip}
	\section{Mon, Jan 23}
		\subsection{Reading}
\lipsum[3-4]

	\Needspace{10\baselineskip}
	\section{Tue, Jan 24}
		\subsection{Reading}
\lipsum[3-4]

	\Needspace{10\baselineskip}
	\section{Wed, Jan 25}
		\subsection{Reading}
\lipsum[3-4]

	\Needspace{10\baselineskip}
	\section{Thu, Jan 26}
		\subsection{Reading}
\lipsum[3-4]

	\Needspace{10\baselineskip}
	\section{Fri, Jan 27}
		\subsection{Reading}
\lipsum[3-4]

	\Needspace{10\baselineskip}
	\section{Sat, Jan 28}
		\subsection{Reading}
\lipsum[3-4]

	\Needspace{10\baselineskip}
	\section{Sun, Jan 29}
		\subsection{Reading}
\lipsum[3-4]

	\Needspace{10\baselineskip}
	\section{Mon, Jan 30}
		\subsection{Reading}
\lipsum[3-4]

	\Needspace{10\baselineskip}
	\section{Tue, Jan 31}
		\subsection{Reading}
\lipsum[3-4]

}{}
\IfFileExists{./Diary_2023/February/leisure_summary.tex}{\chapter{February}
	\Needspace{10\baselineskip}
	\section{Wed, Feb 01}
		\subsection{Reading}
\lipsum[3-4]

	\Needspace{10\baselineskip}
	\section{Thu, Feb 02}
		\subsection{Reading}
\lipsum[3-4]

	\Needspace{10\baselineskip}
	\section{Fri, Feb 03}
		\subsection{Reading}
\lipsum[3-4]

	\Needspace{10\baselineskip}
	\section{Sat, Feb 04}
		\subsection{Reading}
\lipsum[3-4]

	\Needspace{10\baselineskip}
	\section{Sun, Feb 05}
		\subsection{Reading}
\lipsum[3-4]

	\Needspace{10\baselineskip}
	\section{Mon, Feb 06}
		\subsection{Reading}
\lipsum[3-4]

	\Needspace{10\baselineskip}
	\section{Tue, Feb 07}
		\subsection{Reading}
\lipsum[3-4]

	\Needspace{10\baselineskip}
	\section{Wed, Feb 08}
		\subsection{Reading}
\lipsum[3-4]

	\Needspace{10\baselineskip}
	\section{Thu, Feb 09}
		\subsection{Reading}
\lipsum[3-4]

	\Needspace{10\baselineskip}
	\section{Fri, Feb 10}
		\subsection{Reading}
\lipsum[3-4]

	\Needspace{10\baselineskip}
	\section{Sat, Feb 11}
		\subsection{Reading}
\lipsum[3-4]

	\Needspace{10\baselineskip}
	\section{Sun, Feb 12}
		\subsection{Reading}
\lipsum[3-4]

	\Needspace{10\baselineskip}
	\section{Mon, Feb 13}
		\subsection{Reading}
\lipsum[3-4]

	\Needspace{10\baselineskip}
	\section{Tue, Feb 14}
		\subsection{Reading}
\lipsum[3-4]

	\Needspace{10\baselineskip}
	\section{Wed, Feb 15}
		\subsection{Reading}
\lipsum[3-4]

	\Needspace{10\baselineskip}
	\section{Thu, Feb 16}
		\subsection{Reading}
\lipsum[3-4]

	\Needspace{10\baselineskip}
	\section{Fri, Feb 17}
		\subsection{Reading}
\lipsum[3-4]

	\Needspace{10\baselineskip}
	\section{Sat, Feb 18}
		\subsection{Reading}
\lipsum[3-4]

	\Needspace{10\baselineskip}
	\section{Sun, Feb 19}
		\subsection{Reading}
\lipsum[3-4]

	\Needspace{10\baselineskip}
	\section{Mon, Feb 20}
		\subsection{Reading}
\lipsum[3-4]

	\Needspace{10\baselineskip}
	\section{Tue, Feb 21}
		\subsection{Reading}
\lipsum[3-4]

	\Needspace{10\baselineskip}
	\section{Wed, Feb 22}
		\subsection{Reading}
\lipsum[3-4]

	\Needspace{10\baselineskip}
	\section{Thu, Feb 23}
		\subsection{Reading}
\lipsum[3-4]

	\Needspace{10\baselineskip}
	\section{Fri, Feb 24}
		\subsection{Reading}
\lipsum[3-4]

	\Needspace{10\baselineskip}
	\section{Sat, Feb 25}
		\subsection{Reading}
\lipsum[3-4]

	\Needspace{10\baselineskip}
	\section{Sun, Feb 26}
		\subsection{Reading}
\lipsum[3-4]

	\Needspace{10\baselineskip}
	\section{Mon, Feb 27}
		\subsection{Reading}
\lipsum[3-4]

	\Needspace{10\baselineskip}
	\section{Tue, Feb 28}
		\subsection{Reading}
\lipsum[3-4]

}{}
\IfFileExists{./Diary_2023/March/leisure_summary.tex}{\chapter{March}
	\Needspace{10\baselineskip}
	\section{Wed, Mar 01}
		\subsection{Reading}
\lipsum[3-4]

	\Needspace{10\baselineskip}
	\section{Thu, Mar 02}
		\subsection{Reading}
\lipsum[3-4]

	\Needspace{10\baselineskip}
	\section{Fri, Mar 03}
		\subsection{Reading}
\lipsum[3-4]

	\Needspace{10\baselineskip}
	\section{Sat, Mar 04}
		\subsection{Reading}
\lipsum[3-4]

	\Needspace{10\baselineskip}
	\section{Sun, Mar 05}
		\subsection{Reading}
\lipsum[3-4]

	\Needspace{10\baselineskip}
	\section{Mon, Mar 06}
		\subsection{Reading}
\lipsum[3-4]

	\Needspace{10\baselineskip}
	\section{Tue, Mar 07}
		\subsection{Reading}
\lipsum[3-4]

	\Needspace{10\baselineskip}
	\section{Wed, Mar 08}
		\subsection{Reading}
\lipsum[3-4]

	\Needspace{10\baselineskip}
	\section{Thu, Mar 09}
		\subsection{Reading}
\lipsum[3-4]

	\Needspace{10\baselineskip}
	\section{Fri, Mar 10}
		\subsection{Reading}
\lipsum[3-4]

	\Needspace{10\baselineskip}
	\section{Sat, Mar 11}
		\subsection{Reading}
\lipsum[3-4]

	\Needspace{10\baselineskip}
	\section{Sun, Mar 12}
		\subsection{Reading}
\lipsum[3-4]

	\Needspace{10\baselineskip}
	\section{Mon, Mar 13}
		\subsection{Reading}
\lipsum[3-4]

	\Needspace{10\baselineskip}
	\section{Tue, Mar 14}
		\subsection{Reading}
\lipsum[3-4]

	\Needspace{10\baselineskip}
	\section{Wed, Mar 15}
		\subsection{Reading}
\lipsum[3-4]

	\Needspace{10\baselineskip}
	\section{Thu, Mar 16}
		\subsection{Reading}
\lipsum[3-4]

	\Needspace{10\baselineskip}
	\section{Fri, Mar 17}
		\subsection{Reading}
\lipsum[3-4]

	\Needspace{10\baselineskip}
	\section{Sat, Mar 18}
		\subsection{Reading}
\lipsum[3-4]

	\Needspace{10\baselineskip}
	\section{Sun, Mar 19}
		\subsection{Reading}
\lipsum[3-4]

	\Needspace{10\baselineskip}
	\section{Mon, Mar 20}
		\subsection{Reading}
\lipsum[3-4]

	\Needspace{10\baselineskip}
	\section{Tue, Mar 21}
		\subsection{Reading}
\lipsum[3-4]

	\Needspace{10\baselineskip}
	\section{Wed, Mar 22}
		\subsection{Reading}
\lipsum[3-4]

	\Needspace{10\baselineskip}
	\section{Thu, Mar 23}
		\subsection{Reading}
\lipsum[3-4]

	\Needspace{10\baselineskip}
	\section{Fri, Mar 24}
		\subsection{Reading}
\lipsum[3-4]

	\Needspace{10\baselineskip}
	\section{Sat, Mar 25}
		\subsection{Reading}
\lipsum[3-4]

	\Needspace{10\baselineskip}
	\section{Sun, Mar 26}
		\subsection{Reading}
\lipsum[3-4]

	\Needspace{10\baselineskip}
	\section{Mon, Mar 27}
		\subsection{Reading}
\lipsum[3-4]

	\Needspace{10\baselineskip}
	\section{Tue, Mar 28}
		\subsection{Reading}
\lipsum[3-4]

	\Needspace{10\baselineskip}
	\section{Wed, Mar 29}
		\subsection{Reading}
\lipsum[3-4]

	\Needspace{10\baselineskip}
	\section{Thu, Mar 30}
		\subsection{Reading}
\lipsum[3-4]

	\Needspace{10\baselineskip}
	\section{Fri, Mar 31}
		\subsection{Reading}
\lipsum[3-4]

}{}
\IfFileExists{./Diary_2023/April/leisure_summary.tex}{\chapter{April}
	\Needspace{10\baselineskip}
	\section{Sat, Apr 01}
		\subsection{Reading}
\lipsum[3-4]

	\Needspace{10\baselineskip}
	\section{Sun, Apr 02}
		\subsection{Reading}
\lipsum[3-4]

	\Needspace{10\baselineskip}
	\section{Mon, Apr 03}
		\subsection{Reading}
\lipsum[3-4]

	\Needspace{10\baselineskip}
	\section{Tue, Apr 04}
		\subsection{Reading}
\lipsum[3-4]

	\Needspace{10\baselineskip}
	\section{Wed, Apr 05}
		\subsection{Reading}
\lipsum[3-4]

	\Needspace{10\baselineskip}
	\section{Thu, Apr 06}
		\subsection{Reading}
\lipsum[3-4]

	\Needspace{10\baselineskip}
	\section{Fri, Apr 07}
		\subsection{Reading}
\lipsum[3-4]

	\Needspace{10\baselineskip}
	\section{Sat, Apr 08}
		\subsection{Reading}
\lipsum[3-4]

	\Needspace{10\baselineskip}
	\section{Sun, Apr 09}
		\subsection{Reading}
\lipsum[3-4]

	\Needspace{10\baselineskip}
	\section{Mon, Apr 10}
		\subsection{Reading}
\lipsum[3-4]

	\Needspace{10\baselineskip}
	\section{Tue, Apr 11}
		\subsection{Reading}
\lipsum[3-4]

	\Needspace{10\baselineskip}
	\section{Wed, Apr 12}
		\subsection{Reading}
\lipsum[3-4]

	\Needspace{10\baselineskip}
	\section{Thu, Apr 13}
		\subsection{Reading}
\lipsum[3-4]

	\Needspace{10\baselineskip}
	\section{Fri, Apr 14}
		\subsection{Reading}
\lipsum[3-4]

	\Needspace{10\baselineskip}
	\section{Sat, Apr 15}
		\subsection{Reading}
\lipsum[3-4]

	\Needspace{10\baselineskip}
	\section{Sun, Apr 16}
		\subsection{Reading}
\lipsum[3-4]

	\Needspace{10\baselineskip}
	\section{Mon, Apr 17}
		\subsection{Reading}
\lipsum[3-4]

	\Needspace{10\baselineskip}
	\section{Tue, Apr 18}
		\subsection{Reading}
\lipsum[3-4]

	\Needspace{10\baselineskip}
	\section{Wed, Apr 19}
		\subsection{Reading}
\lipsum[3-4]

	\Needspace{10\baselineskip}
	\section{Thu, Apr 20}
		\subsection{Reading}
\lipsum[3-4]

	\Needspace{10\baselineskip}
	\section{Fri, Apr 21}
		\subsection{Reading}
\lipsum[3-4]

	\Needspace{10\baselineskip}
	\section{Sat, Apr 22}
		\subsection{Reading}
\lipsum[3-4]

	\Needspace{10\baselineskip}
	\section{Sun, Apr 23}
		\subsection{Reading}
\lipsum[3-4]

	\Needspace{10\baselineskip}
	\section{Mon, Apr 24}
		\subsection{Reading}
\lipsum[3-4]

	\Needspace{10\baselineskip}
	\section{Tue, Apr 25}
		\subsection{Reading}
\lipsum[3-4]

	\Needspace{10\baselineskip}
	\section{Wed, Apr 26}
		\subsection{Reading}
\lipsum[3-4]

	\Needspace{10\baselineskip}
	\section{Thu, Apr 27}
		\subsection{Reading}
\lipsum[3-4]

	\Needspace{10\baselineskip}
	\section{Fri, Apr 28}
		\subsection{Reading}
\lipsum[3-4]

	\Needspace{10\baselineskip}
	\section{Sat, Apr 29}
		\subsection{Reading}
\lipsum[3-4]

	\Needspace{10\baselineskip}
	\section{Sun, Apr 30}
		\subsection{Reading}
\lipsum[3-4]

}{}
\IfFileExists{./Diary_2023/May/leisure_summary.tex}{\chapter{May}
	\Needspace{10\baselineskip}
	\section{Mon, May 01}
		\subsection{Reading}
\lipsum[3-4]

	\Needspace{10\baselineskip}
	\section{Tue, May 02}
		\subsection{Reading}
\lipsum[3-4]

	\Needspace{10\baselineskip}
	\section{Wed, May 03}
		\subsection{Reading}
\lipsum[3-4]

	\Needspace{10\baselineskip}
	\section{Thu, May 04}
		\subsection{Reading}
\lipsum[3-4]

	\Needspace{10\baselineskip}
	\section{Fri, May 05}
		\subsection{Reading}
\lipsum[3-4]

	\Needspace{10\baselineskip}
	\section{Sat, May 06}
		\subsection{Reading}
\lipsum[3-4]

	\Needspace{10\baselineskip}
	\section{Sun, May 07}
		\subsection{Reading}
\lipsum[3-4]

	\Needspace{10\baselineskip}
	\section{Mon, May 08}
		\subsection{Reading}
\lipsum[3-4]

	\Needspace{10\baselineskip}
	\section{Tue, May 09}
		\subsection{Reading}
\lipsum[3-4]

	\Needspace{10\baselineskip}
	\section{Wed, May 10}
		\subsection{Reading}
\lipsum[3-4]

	\Needspace{10\baselineskip}
	\section{Thu, May 11}
		\subsection{Reading}
\lipsum[3-4]

	\Needspace{10\baselineskip}
	\section{Fri, May 12}
		\subsection{Reading}
\lipsum[3-4]

	\Needspace{10\baselineskip}
	\section{Sat, May 13}
		\subsection{Reading}
\lipsum[3-4]

	\Needspace{10\baselineskip}
	\section{Sun, May 14}
		\subsection{Reading}
\lipsum[3-4]

	\Needspace{10\baselineskip}
	\section{Mon, May 15}
		\subsection{Reading}
\lipsum[3-4]

	\Needspace{10\baselineskip}
	\section{Tue, May 16}
		\subsection{Reading}
\lipsum[3-4]

	\Needspace{10\baselineskip}
	\section{Wed, May 17}
		\subsection{Reading}
\lipsum[3-4]

	\Needspace{10\baselineskip}
	\section{Thu, May 18}
		\subsection{Reading}
\lipsum[3-4]

	\Needspace{10\baselineskip}
	\section{Fri, May 19}
		\subsection{Reading}
\lipsum[3-4]

	\Needspace{10\baselineskip}
	\section{Sat, May 20}
		\subsection{Reading}
\lipsum[3-4]

	\Needspace{10\baselineskip}
	\section{Sun, May 21}
		\subsection{Reading}
\lipsum[3-4]

	\Needspace{10\baselineskip}
	\section{Mon, May 22}
		\subsection{Reading}
\lipsum[3-4]

	\Needspace{10\baselineskip}
	\section{Tue, May 23}
		\subsection{Reading}
\lipsum[3-4]

	\Needspace{10\baselineskip}
	\section{Wed, May 24}
		\subsection{Reading}
\lipsum[3-4]

	\Needspace{10\baselineskip}
	\section{Thu, May 25}
		\subsection{Reading}
\lipsum[3-4]

	\Needspace{10\baselineskip}
	\section{Fri, May 26}
		\subsection{Reading}
\lipsum[3-4]

	\Needspace{10\baselineskip}
	\section{Sat, May 27}
		\subsection{Reading}
\lipsum[3-4]

	\Needspace{10\baselineskip}
	\section{Sun, May 28}
		\subsection{Reading}
\lipsum[3-4]

	\Needspace{10\baselineskip}
	\section{Mon, May 29}
		\subsection{Reading}
\lipsum[3-4]

	\Needspace{10\baselineskip}
	\section{Tue, May 30}
		\subsection{Reading}
\lipsum[3-4]

	\Needspace{10\baselineskip}
	\section{Wed, May 31}
		\subsection{Reading}
\lipsum[3-4]

}{}
\IfFileExists{./Diary_2023/June/leisure_summary.tex}{\chapter{June}
	\Needspace{10\baselineskip}
	\section{Thu, Jun 01}
		\subsection{Reading}
\lipsum[3-4]

	\Needspace{10\baselineskip}
	\section{Fri, Jun 02}
		\subsection{Reading}
\lipsum[3-4]

	\Needspace{10\baselineskip}
	\section{Sat, Jun 03}
		\subsection{Reading}
\lipsum[3-4]

	\Needspace{10\baselineskip}
	\section{Sun, Jun 04}
		\subsection{Reading}
\lipsum[3-4]

	\Needspace{10\baselineskip}
	\section{Mon, Jun 05}
		\subsection{Reading}
\lipsum[3-4]

	\Needspace{10\baselineskip}
	\section{Tue, Jun 06}
		\subsection{Reading}
\lipsum[3-4]

	\Needspace{10\baselineskip}
	\section{Wed, Jun 07}
		\subsection{Reading}
\lipsum[3-4]

	\Needspace{10\baselineskip}
	\section{Thu, Jun 08}
		\subsection{Reading}
\lipsum[3-4]

	\Needspace{10\baselineskip}
	\section{Fri, Jun 09}
		\subsection{Reading}
\lipsum[3-4]

	\Needspace{10\baselineskip}
	\section{Sat, Jun 10}
		\subsection{Reading}
\lipsum[3-4]

	\Needspace{10\baselineskip}
	\section{Sun, Jun 11}
		\subsection{Reading}
\lipsum[3-4]

	\Needspace{10\baselineskip}
	\section{Mon, Jun 12}
		\subsection{Reading}
\lipsum[3-4]

	\Needspace{10\baselineskip}
	\section{Tue, Jun 13}
		\subsection{Reading}
\lipsum[3-4]

	\Needspace{10\baselineskip}
	\section{Wed, Jun 14}
		\subsection{Reading}
\lipsum[3-4]

	\Needspace{10\baselineskip}
	\section{Thu, Jun 15}
		\subsection{Reading}
\lipsum[3-4]

	\Needspace{10\baselineskip}
	\section{Fri, Jun 16}
		\subsection{Reading}
\lipsum[3-4]

	\Needspace{10\baselineskip}
	\section{Sat, Jun 17}
		\subsection{Reading}
\lipsum[3-4]

	\Needspace{10\baselineskip}
	\section{Sun, Jun 18}
		\subsection{Reading}
\lipsum[3-4]

	\Needspace{10\baselineskip}
	\section{Mon, Jun 19}
		\subsection{Reading}
\lipsum[3-4]

	\Needspace{10\baselineskip}
	\section{Tue, Jun 20}
		\subsection{Reading}
\lipsum[3-4]

	\Needspace{10\baselineskip}
	\section{Wed, Jun 21}
		\subsection{Reading}
\lipsum[3-4]

	\Needspace{10\baselineskip}
	\section{Thu, Jun 22}
		\subsection{Reading}
\lipsum[3-4]

	\Needspace{10\baselineskip}
	\section{Fri, Jun 23}
		\subsection{Reading}
\lipsum[3-4]

	\Needspace{10\baselineskip}
	\section{Sat, Jun 24}
		\subsection{Reading}
\lipsum[3-4]

	\Needspace{10\baselineskip}
	\section{Sun, Jun 25}
		\subsection{Reading}
\lipsum[3-4]

	\Needspace{10\baselineskip}
	\section{Mon, Jun 26}
		\subsection{Reading}
\lipsum[3-4]

	\Needspace{10\baselineskip}
	\section{Tue, Jun 27}
		\subsection{Reading}
\lipsum[3-4]

	\Needspace{10\baselineskip}
	\section{Wed, Jun 28}
		\subsection{Reading}
\lipsum[3-4]

	\Needspace{10\baselineskip}
	\section{Thu, Jun 29}
		\subsection{Reading}
\lipsum[3-4]

	\Needspace{10\baselineskip}
	\section{Fri, Jun 30}
		\subsection{Reading}
\lipsum[3-4]

}{}
\IfFileExists{./Diary_2023/July/leisure_summary.tex}{\chapter{July}
	\section{Sat, Jul 01}
		\subsection{Reading}
\lipsum[3-4]

	\section{Sun, Jul 02}
		\subsection{Reading}
\lipsum[3-4]

	\section{Mon, Jul 03}
		\subsection{Reading}
\lipsum[3-4]

	\section{Tue, Jul 04}
		\subsection{Reading}
\lipsum[3-4]

	\section{Wed, Jul 05}
		\subsection{Reading}
\lipsum[3-4]

	\section{Thu, Jul 06}
		\subsection{Reading}
\lipsum[3-4]

	\section{Fri, Jul 07}
		\subsection{Reading}
\lipsum[3-4]

	\section{Sat, Jul 08}
		\subsection{Reading}
\lipsum[3-4]

	\section{Sun, Jul 09}
		\subsection{Reading}
\lipsum[3-4]

	\section{Mon, Jul 10}
		\subsection{Reading}
\lipsum[3-4]

	\section{Tue, Jul 11}
		\subsection{Reading}
\lipsum[3-4]

	\section{Wed, Jul 12}
		\subsection{Reading}
\lipsum[3-4]

	\section{Thu, Jul 13}
		\subsection{Reading}
\lipsum[3-4]

	\section{Fri, Jul 14}
		\subsection{Reading}
\lipsum[3-4]

	\section{Sat, Jul 15}
		\subsection{Reading}
\lipsum[3-4]

	\section{Sun, Jul 16}
		\subsection{Reading}
\lipsum[3-4]

	\section{Mon, Jul 17}
		\subsection{Reading}
\lipsum[3-4]

	\section{Tue, Jul 18}
		\subsection{Reading}
\lipsum[3-4]

	\section{Wed, Jul 19}
		\subsection{Reading}
\lipsum[3-4]

	\section{Thu, Jul 20}
		\subsection{Reading}
\lipsum[3-4]

	\section{Fri, Jul 21}
		\subsection{Reading}
\lipsum[3-4]

	\section{Sat, Jul 22}
		\subsection{Reading}
\lipsum[3-4]

	\section{Sun, Jul 23}
		\subsection{Reading}
\lipsum[3-4]

	\section{Mon, Jul 24}
		\subsection{Reading}
\lipsum[3-4]

	\section{Tue, Jul 25}
		\subsection{Reading}
\lipsum[3-4]

	\section{Wed, Jul 26}
		\subsection{Reading}
\lipsum[3-4]

	\section{Thu, Jul 27}
		\subsection{Reading}
\lipsum[3-4]

	\section{Fri, Jul 28}
		\subsection{Reading}
\lipsum[3-4]

	\section{Sat, Jul 29}
		\subsection{Reading}
\lipsum[3-4]

	\section{Sun, Jul 30}
		\subsection{Reading}
\lipsum[3-4]

	\section{Mon, Jul 31}
		\subsection{Reading}
\lipsum[3-4]

}{}
\IfFileExists{./Diary_2023/August/leisure_summary.tex}{\chapter{August}
	\section{Tue, Aug 01}
		\subsection{Reading}
\lipsum[3-4]

	\section{Wed, Aug 02}
		\subsection{Reading}
\lipsum[3-4]

	\section{Thu, Aug 03}
		\subsection{Reading}
\lipsum[3-4]

	\section{Fri, Aug 04}
		\subsection{Reading}
\lipsum[3-4]

	\section{Sat, Aug 05}
		\subsection{Reading}
\lipsum[3-4]

	\section{Sun, Aug 06}
		\subsection{Reading}
\lipsum[3-4]

	\section{Mon, Aug 07}
		\subsection{Reading}
\lipsum[3-4]

	\section{Tue, Aug 08}
		\subsection{Reading}
\lipsum[3-4]

	\section{Wed, Aug 09}
		\subsection{Reading}
\lipsum[3-4]

	\section{Thu, Aug 10}
		\subsection{Reading}
\lipsum[3-4]

	\section{Fri, Aug 11}
		\subsection{Reading}
\lipsum[3-4]

	\section{Sat, Aug 12}
		\subsection{Reading}
\lipsum[3-4]

	\section{Sun, Aug 13}
		\subsection{Reading}
\lipsum[3-4]

	\section{Mon, Aug 14}
		\subsection{Reading}
\lipsum[3-4]

	\section{Tue, Aug 15}
		\subsection{Reading}
\lipsum[3-4]

	\section{Wed, Aug 16}
		\subsection{Reading}
\lipsum[3-4]

	\section{Thu, Aug 17}
		\subsection{Reading}
\lipsum[3-4]

	\section{Fri, Aug 18}
		\subsection{Reading}
\lipsum[3-4]

	\section{Sat, Aug 19}
		\subsection{Reading}
\lipsum[3-4]

	\section{Sun, Aug 20}
		\subsection{Reading}
\lipsum[3-4]

	\section{Mon, Aug 21}
		\subsection{Reading}
\lipsum[3-4]

	\section{Tue, Aug 22}
		\subsection{Reading}
\lipsum[3-4]

	\section{Wed, Aug 23}
		\subsection{Reading}
\lipsum[3-4]

	\section{Thu, Aug 24}
		\subsection{Reading}
\lipsum[3-4]

	\section{Fri, Aug 25}
		\subsection{Reading}
\lipsum[3-4]

	\section{Sat, Aug 26}
		\subsection{Reading}
\lipsum[3-4]

	\section{Sun, Aug 27}
		\subsection{Reading}
\lipsum[3-4]

	\section{Mon, Aug 28}
		\subsection{Reading}
\lipsum[3-4]

	\section{Tue, Aug 29}
		\subsection{Reading}
\lipsum[3-4]

	\section{Wed, Aug 30}
		\subsection{Reading}
\lipsum[3-4]

	\section{Thu, Aug 31}
		\subsection{Reading}
\lipsum[3-4]

}{}
\IfFileExists{./Diary_2023/September/leisure_summary.tex}{\chapter{July}
	\section{Sat, Jul 01}
		\subsection{Reading}
\lipsum[3-4]

	\section{Sun, Jul 02}
		\subsection{Reading}
\lipsum[3-4]

	\section{Mon, Jul 03}
		\subsection{Reading}
\lipsum[3-4]

	\section{Tue, Jul 04}
		\subsection{Reading}
\lipsum[3-4]

	\section{Wed, Jul 05}
		\subsection{Reading}
\lipsum[3-4]

	\section{Thu, Jul 06}
		\subsection{Reading}
\lipsum[3-4]

	\section{Fri, Jul 07}
		\subsection{Reading}
\lipsum[3-4]

	\section{Sat, Jul 08}
		\subsection{Reading}
\lipsum[3-4]

	\section{Sun, Jul 09}
		\subsection{Reading}
\lipsum[3-4]

	\section{Mon, Jul 10}
		\subsection{Reading}
\lipsum[3-4]

	\section{Tue, Jul 11}
		\subsection{Reading}
\lipsum[3-4]

	\section{Wed, Jul 12}
		\subsection{Reading}
\lipsum[3-4]

	\section{Thu, Jul 13}
		\subsection{Reading}
\lipsum[3-4]

	\section{Fri, Jul 14}
		\subsection{Reading}
\lipsum[3-4]

	\section{Sat, Jul 15}
		\subsection{Reading}
\lipsum[3-4]

	\section{Sun, Jul 16}
		\subsection{Reading}
\lipsum[3-4]

	\section{Mon, Jul 17}
		\subsection{Reading}
\lipsum[3-4]

	\section{Tue, Jul 18}
		\subsection{Reading}
\lipsum[3-4]

	\section{Wed, Jul 19}
		\subsection{Reading}
\lipsum[3-4]

	\section{Thu, Jul 20}
		\subsection{Reading}
\lipsum[3-4]

	\section{Fri, Jul 21}
		\subsection{Reading}
\lipsum[3-4]

	\section{Sat, Jul 22}
		\subsection{Reading}
\lipsum[3-4]

	\section{Sun, Jul 23}
		\subsection{Reading}
\lipsum[3-4]

	\section{Mon, Jul 24}
		\subsection{Reading}
\lipsum[3-4]

	\section{Tue, Jul 25}
		\subsection{Reading}
\lipsum[3-4]

	\section{Wed, Jul 26}
		\subsection{Reading}
\lipsum[3-4]

	\section{Thu, Jul 27}
		\subsection{Reading}
\lipsum[3-4]

	\section{Fri, Jul 28}
		\subsection{Reading}
\lipsum[3-4]

	\section{Sat, Jul 29}
		\subsection{Reading}
\lipsum[3-4]

	\section{Sun, Jul 30}
		\subsection{Reading}
\lipsum[3-4]

	\section{Mon, Jul 31}
		\subsection{Reading}
\lipsum[3-4]

}{}
\IfFileExists{./Diary_2023/October/leisure_summary.tex}{\chapter{October}
	\section{Sun, Oct 01}
		\subsection{Reading}
\lipsum[3-4]

	\section{Mon, Oct 02}
		\subsection{Reading}
\lipsum[3-4]

	\section{Tue, Oct 03}
		\subsection{Reading}
\lipsum[3-4]

	\section{Wed, Oct 04}
		\subsection{Reading}
\lipsum[3-4]

	\section{Thu, Oct 05}
		\subsection{Reading}
\lipsum[3-4]

	\section{Fri, Oct 06}
		\subsection{Reading}
\lipsum[3-4]

	\section{Sat, Oct 07}
		\subsection{Reading}
\lipsum[3-4]

	\section{Sun, Oct 08}
		\subsection{Reading}
\lipsum[3-4]

	\section{Mon, Oct 09}
		\subsection{Reading}
\lipsum[3-4]

	\section{Tue, Oct 10}
		\subsection{Reading}
\lipsum[3-4]

	\section{Wed, Oct 11}
		\subsection{Reading}
\lipsum[3-4]

	\section{Thu, Oct 12}
		\subsection{Reading}
\lipsum[3-4]

	\section{Fri, Oct 13}
		\subsection{Reading}
\lipsum[3-4]

	\section{Sat, Oct 14}
		\subsection{Reading}
\lipsum[3-4]

	\section{Sun, Oct 15}
		\subsection{Reading}
\lipsum[3-4]

	\section{Mon, Oct 16}
		\subsection{Reading}
\lipsum[3-4]

	\section{Tue, Oct 17}
		\subsection{Reading}
\lipsum[3-4]

	\section{Wed, Oct 18}
		\subsection{Reading}
\lipsum[3-4]

	\section{Thu, Oct 19}
		\subsection{Reading}
\lipsum[3-4]

	\section{Fri, Oct 20}
		\subsection{Reading}
\lipsum[3-4]

	\section{Sat, Oct 21}
		\subsection{Reading}
\lipsum[3-4]

	\section{Sun, Oct 22}
		\subsection{Reading}
\lipsum[3-4]

	\section{Mon, Oct 23}
		\subsection{Reading}
\lipsum[3-4]

	\section{Tue, Oct 24}
		\subsection{Reading}
\lipsum[3-4]

	\section{Wed, Oct 25}
		\subsection{Reading}
\lipsum[3-4]

	\section{Thu, Oct 26}
		\subsection{Reading}
\lipsum[3-4]

	\section{Fri, Oct 27}
		\subsection{Reading}
\lipsum[3-4]

	\section{Sat, Oct 28}
		\subsection{Reading}
\lipsum[3-4]

	\section{Sun, Oct 29}
		\subsection{Reading}
\lipsum[3-4]

	\section{Mon, Oct 30}
		\subsection{Reading}
\lipsum[3-4]

	\section{Tue, Oct 31}
		\subsection{Reading}
\lipsum[3-4]

}{}
\IfFileExists{./Diary_2023/November/leisure_summary.tex}{\chapter{November}
	\Needspace{10\baselineskip}
	\section{Wed, Nov 01}
		\subsection{Reading}
\lipsum[3-4]

	\Needspace{10\baselineskip}
	\section{Thu, Nov 02}
		\subsection{Reading}
\lipsum[3-4]

	\Needspace{10\baselineskip}
	\section{Fri, Nov 03}
		\subsection{Reading}
\lipsum[3-4]

	\Needspace{10\baselineskip}
	\section{Sat, Nov 04}
		\subsection{Reading}
\lipsum[3-4]

	\Needspace{10\baselineskip}
	\section{Sun, Nov 05}
		\subsection{Reading}
\lipsum[3-4]

	\Needspace{10\baselineskip}
	\section{Mon, Nov 06}
		\subsection{Reading}
\lipsum[3-4]

	\Needspace{10\baselineskip}
	\section{Tue, Nov 07}
		\subsection{Reading}
\lipsum[3-4]

	\Needspace{10\baselineskip}
	\section{Wed, Nov 08}
		\subsection{Reading}
\lipsum[3-4]

	\Needspace{10\baselineskip}
	\section{Thu, Nov 09}
		\subsection{Reading}
\lipsum[3-4]

	\Needspace{10\baselineskip}
	\section{Fri, Nov 10}
		\subsection{Reading}
\lipsum[3-4]

	\Needspace{10\baselineskip}
	\section{Sat, Nov 11}
		\subsection{Reading}
\lipsum[3-4]

	\Needspace{10\baselineskip}
	\section{Sun, Nov 12}
		\subsection{Reading}
\lipsum[3-4]

	\Needspace{10\baselineskip}
	\section{Mon, Nov 13}
		\subsection{Reading}
\lipsum[3-4]

	\Needspace{10\baselineskip}
	\section{Tue, Nov 14}
		\subsection{Reading}
\lipsum[3-4]

	\Needspace{10\baselineskip}
	\section{Wed, Nov 15}
		\subsection{Reading}
\lipsum[3-4]

	\Needspace{10\baselineskip}
	\section{Thu, Nov 16}
		\subsection{Reading}
\lipsum[3-4]

	\Needspace{10\baselineskip}
	\section{Fri, Nov 17}
		\subsection{Reading}
\lipsum[3-4]

	\Needspace{10\baselineskip}
	\section{Sat, Nov 18}
		\subsection{Reading}
\lipsum[3-4]

	\Needspace{10\baselineskip}
	\section{Sun, Nov 19}
		\subsection{Reading}
\lipsum[3-4]

	\Needspace{10\baselineskip}
	\section{Mon, Nov 20}
		\subsection{Reading}
\lipsum[3-4]

	\Needspace{10\baselineskip}
	\section{Tue, Nov 21}
		\subsection{Reading}
\lipsum[3-4]

	\Needspace{10\baselineskip}
	\section{Wed, Nov 22}
		\subsection{Reading}
\lipsum[3-4]

	\Needspace{10\baselineskip}
	\section{Thu, Nov 23}
		\subsection{Reading}
\lipsum[3-4]

	\Needspace{10\baselineskip}
	\section{Fri, Nov 24}
		\subsection{Reading}
\lipsum[3-4]

	\Needspace{10\baselineskip}
	\section{Sat, Nov 25}
		\subsection{Reading}
\lipsum[3-4]

	\Needspace{10\baselineskip}
	\section{Sun, Nov 26}
		\subsection{Reading}
\lipsum[3-4]

	\Needspace{10\baselineskip}
	\section{Mon, Nov 27}
		\subsection{Reading}
\lipsum[3-4]

	\Needspace{10\baselineskip}
	\section{Tue, Nov 28}
		\subsection{Reading}
\lipsum[3-4]

	\Needspace{10\baselineskip}
	\section{Wed, Nov 29}
		\subsection{Reading}
\lipsum[3-4]

	\Needspace{10\baselineskip}
	\section{Thu, Nov 30}
		\subsection{Reading}
\lipsum[3-4]

}{}
\IfFileExists{./Diary_2023/December/leisure_summary.tex}{\chapter{July}
	\section{Sat, Jul 01}
		\subsection{Reading}
\lipsum[3-4]

	\section{Sun, Jul 02}
		\subsection{Reading}
\lipsum[3-4]

	\section{Mon, Jul 03}
		\subsection{Reading}
\lipsum[3-4]

	\section{Tue, Jul 04}
		\subsection{Reading}
\lipsum[3-4]

	\section{Wed, Jul 05}
		\subsection{Reading}
\lipsum[3-4]

	\section{Thu, Jul 06}
		\subsection{Reading}
\lipsum[3-4]

	\section{Fri, Jul 07}
		\subsection{Reading}
\lipsum[3-4]

	\section{Sat, Jul 08}
		\subsection{Reading}
\lipsum[3-4]

	\section{Sun, Jul 09}
		\subsection{Reading}
\lipsum[3-4]

	\section{Mon, Jul 10}
		\subsection{Reading}
\lipsum[3-4]

	\section{Tue, Jul 11}
		\subsection{Reading}
\lipsum[3-4]

	\section{Wed, Jul 12}
		\subsection{Reading}
\lipsum[3-4]

	\section{Thu, Jul 13}
		\subsection{Reading}
\lipsum[3-4]

	\section{Fri, Jul 14}
		\subsection{Reading}
\lipsum[3-4]

	\section{Sat, Jul 15}
		\subsection{Reading}
\lipsum[3-4]

	\section{Sun, Jul 16}
		\subsection{Reading}
\lipsum[3-4]

	\section{Mon, Jul 17}
		\subsection{Reading}
\lipsum[3-4]

	\section{Tue, Jul 18}
		\subsection{Reading}
\lipsum[3-4]

	\section{Wed, Jul 19}
		\subsection{Reading}
\lipsum[3-4]

	\section{Thu, Jul 20}
		\subsection{Reading}
\lipsum[3-4]

	\section{Fri, Jul 21}
		\subsection{Reading}
\lipsum[3-4]

	\section{Sat, Jul 22}
		\subsection{Reading}
\lipsum[3-4]

	\section{Sun, Jul 23}
		\subsection{Reading}
\lipsum[3-4]

	\section{Mon, Jul 24}
		\subsection{Reading}
\lipsum[3-4]

	\section{Tue, Jul 25}
		\subsection{Reading}
\lipsum[3-4]

	\section{Wed, Jul 26}
		\subsection{Reading}
\lipsum[3-4]

	\section{Thu, Jul 27}
		\subsection{Reading}
\lipsum[3-4]

	\section{Fri, Jul 28}
		\subsection{Reading}
\lipsum[3-4]

	\section{Sat, Jul 29}
		\subsection{Reading}
\lipsum[3-4]

	\section{Sun, Jul 30}
		\subsection{Reading}
\lipsum[3-4]

	\section{Mon, Jul 31}
		\subsection{Reading}
\lipsum[3-4]

}{}


%%------------------------ BIBLIOGRAPHY ----------------------------%%

\bibliographystyle{apalike}
\bibliography{bibliography}

\end{document}